%-----------------------------------
% Define document and include general packages
%-----------------------------------
\documentclass[12pt,oneside,titlepage,listof=totoc,bibliography=totoc]{scrartcl}
\usepackage[utf8]{inputenc}
\usepackage[ngerman]{babel}
\usepackage[babel,german=quotes]{csquotes}
\usepackage[T1]{fontenc}
\usepackage{fancyhdr}
\usepackage{fancybox}
\usepackage[a4paper,left=4cm,right=2cm,top=1.5cm,bottom=2cm,includeheadfoot]{geometry}
\usepackage{graphicx}
\usepackage{colortbl}
\usepackage{array}
\usepackage{float}      %Positionierung von Abb. und Tabellen mit [H] erzwingen
\usepackage{footnote}
\usepackage{caption}
\usepackage{mdwlist}
\usepackage{amssymb}
\usepackage{mathptmx}
\usepackage{amsmath}
\usepackage[table]{xcolor}
\usepackage{marvosym}			% Verwendung von Symbolen, z.B. perfektes Eurozeichen
\usepackage[colorlinks=true,linkcolor=black]{hyperref}
\definecolor{darkblack}{rgb}{0,0,0}
\hypersetup{colorlinks=true, breaklinks=true, linkcolor=darkblack, menucolor=darkblack, urlcolor=darkblack}
\usepackage{times}
\fontfamily{ptm}\selectfont

%Pakete für Tabellen
\usepackage{epstopdf}
\usepackage{nicefrac} % Brüche
\usepackage{multirow}
\usepackage{rotating} % vertikal schreiben
\usepackage{colortbl}
\usepackage{mdwlist}

\definecolor{dunkelgrau}{rgb}{0.8,0.8,0.8}
\definecolor{hellgrau}{rgb}{0.0,0.7,0.99}
% Colors for listings
\definecolor{mauve}{rgb}{0.58,0,0.82}
\definecolor{dkgreen}{rgb}{0,0.6,0}

% sauber formatierter Quelltext
\usepackage{listings}
\lstset{numbers=left,
	numberstyle=\tiny,
	numbersep=5pt,
	breaklines=true,
	showstringspaces=false,
	frame=l ,
	xleftmargin=5pt,
	xrightmargin=5pt,
	basicstyle=\ttfamily\scriptsize,
	stepnumber=1,
	keywordstyle=\color{blue},          % keyword style
  	commentstyle=\color{dkgreen},       % comment style
  	stringstyle=\color{mauve}         % string literal style
}

% Biblatex
\usepackage[
backend=biber,
style=numeric,
citestyle=authoryear,
url=false,
isbn=false,
notetype=footonly,
hyperref=false,
sortlocale=de]{biblatex}

%weitere Anpassungen für BibLaTex
% Optionen für Biblatex
% Laut Leitfaden:
% - ist es besser den Vornamen in Kurzform anzugeben
% - Bücher ohne ISBN anzugeben
\ExecuteBibliographyOptions{%
giveninits=true,
isbn=false,
url=true,
doi=false,
eprint=false,
maxbibnames=7, % Alle Autoren (kein et al.)
maxcitenames=2, % et al. ab dem 3. Autor
backref=false, % Rückverweise auf Zitatseiten
bibencoding=utf8, % wenn .bib in utf8, sonst ascii
bibwarn=true, % Warnung bei fehlerhafter bib-Datei
}%

% et al. an Stelle von u.a.
\DefineBibliographyStrings{ngerman}{
   andothers = {{et\,al\adddot}},
}

% Klammern um das Jahr in der Fußnote
\renewbibmacro*{cite:labelyear+extrayear}{%
  \iffieldundef{labelyear}
    {}
    {\printtext[bibhyperref]{%
       \mkbibparens{%
         \printfield{labelyear}%
         \printfield{extrayear}}}}}

\DeclareNameFormat{last-first}{%
  \iffirstinits
    {\usebibmacro{name:family-given}
        {\namepartfamily}
        {\namepartgiveni}
        {\namepartprefix}
        {\namepartsuffix}
    }
    {\usebibmacro{name:family-given}
        {\namepartfamily}
        {\namepartgiven}
        {\namepartprefix}
        {\namepartsuffix}
    }%
  \usebibmacro{name:andothers}}

% Alternative Notation der Fußnoten
% Zeigt sowohl den Nachnamen als auch den Vornamen an
% Beispiel: \fullfootcite[Vgl. ][Seite 5]{Tanenbaum.2003}
\DeclareCiteCommand{\fullfootcite}[\mkbibfootnote]
  {\usebibmacro{prenote}}
  {\usebibmacro{citeindex}%
    \printnames[sortname][1-1]{author}%
    \addspace (\printfield{year})}
  {\addsemicolon\space}
  {\usebibmacro{postnote}}

%Autoren (Nachname, Vorname)
\DeclareNameAlias{default}{family-given}

% Laut Leitfaden sind:
% - die Titel (egal welche) ohne Anfuehrungszeichen
% - Anpassung von @Inbook elementen
% - Anpassung von Online Quellen
\DeclareFieldFormat[article,book,inbook,inproceedings]{title}{{#1}}
\DeclareFieldFormat[inproceedings]{booktitle}{{#1}}
\DeclareFieldFormat[inbook]{chapter}{{#1}}

% Laut Leitfaden...
\DeclareFieldFormat{urldate}{\addcomma\space\bibstring{urlseen}\space#1}
\DefineBibliographyStrings{german}{%
	urlseen = {Abruf am},
}

% Reihenfolge von publisher, year, address verändern

% Definiert @Book Eintrag
% Laut Leitfaden ...
\DeclareBibliographyDriver{book}{%
  \printnames{author}%
  \setunit*{\space (}%
  \printfield{year}\newunit{)}%
  \newunit\addcolon\space
  \printfield{title}%
  \setunit*{,\space}%
  \printlist{location}%
  \setunit*{\space}%
  \printfield{year}%
  \finentry}

% Add @Inbook style
% Laut Leitfaden ...
\DeclareBibliographyDriver{inbook}{%
  \printnames{author}%
  \setunit*{\space (}%
  \printfield{year}\newunit{)}%
  \newunit\addcolon\space
  \printfield{chapter}%
  \setunit*{, in:\space}%
  \printnames{editor}\newunit{\space (Hrsg.)}%
  \setunit*{, \space}%
  \printfield{title}%
  \setunit*{,\space}%
  \printlist{location}%
  \setunit*{\space}%
  \printfield{year}%
  \setunit*{, \space}%
  \printfield{pages}%
  \finentry}

% Definiert @Online Eintrag
% Laut Leitfaden ...
\DeclareBibliographyDriver{online}{%
  \printnames{author}%
  \newunit\newblockpunct
  \printfield{title}%
  \setunit*{,\space}%
  %\newunit\newblock
  \printfield{url}%
  \setunit*{,\space Erscheinungsjahr:\space}%
  \printfield{year}%
  \setunit*{,\space Aufruf am:\space}%
  \printfield{note}%
  \finentry}

% Definiert @Article Eintrag
% Laut Leitfaden ...
\DeclareBibliographyDriver{article}{%
  \printnames{author}%
  \setunit*{\space (}%
  \printfield{year}\newunit{)}%
  \newunit\addcolon\space
  \printfield{title}%
  \setunit*{.\space In:\space}%
  %\newunit\newblock
  \usebibmacro{journal}%
  \setunit*{\space }%
  \printfield{year}%
  \setunit*{, \space}%
  \setunit*{, Nr. \space}%
  \printfield{number}\newunit{, \space}%
  \printfield{pages}%
  \finentry}

% Definiert @Inproceeedings Eintrag
% (booktitle != title) - das ist Grund zum unterschied @Article
% Laut Leitfaden ...
\DeclareBibliographyDriver{inproceedings}{%
  \printnames{author}%
  \setunit*{\space (}%
  \printfield{year}\newunit{)}%
  \newunit\addcolon\space
  \printfield{title}%
  \setunit*{.\space In:\space}%
  %\newunit\newblock
  \printfield{booktitle}%
  \setunit*{\space }%
  \printfield{year}%
  \setunit*{, \space}%
  \setunit*{, Nr. \space}%
  \printfield{number}\newunit{, \space}%
  \printfield{pages}%
  \finentry}


%Doppelpunkt nach dem letzten Autor
\renewcommand*{\labelnamepunct}{\addcolon\addspace }

%Komma an Stelle des Punktes
\renewcommand*{\newunitpunct}{\addcomma\space}

%Autoren durch Semikolon trennen
\newcommand*{\bibmultinamedelim}{\addsemicolon\space}%
\newcommand*{\bibfinalnamedelim}{\addsemicolon\space}%
\AtBeginBibliography{%
  \let\multinamedelim\bibmultinamedelim
  \let\finalnamedelim\bibfinalnamedelim
}

%Titel nicht kursiv anzeigen
\DeclareFieldFormat{title}{#1\isdot}



%Bib-Datei einbinden
\addbibresource{literatur/literatur.bib}

% Pfad fuer Abbildungen
\graphicspath{{./}{./abbildungen/}}

%-----------------------------------
% Weitere Ebene einfügen
\usepackage{titletoc}

\makeatletter

% Setze die Tiefe des Inhaltsverzeichnis auf 4 Ebenen
% Damit erscheinen \paragraph-Sektionen auch im Inhaltsverzeichnis
\setcounter{secnumdepth}{4}
\setcounter{tocdepth}{4}

% Fuege Abstand nach unten wie in einer normalen \section hinzu
% Andernfalls haette \paragraph keinen Zeilenumbruch
% Der Zeilenumbruch koennte mit einer leeren \mbox{} ersetzt werden
% Jedoch klebt dann der Text relativ nah an der Ueberschrift
\renewcommand{\paragraph}{%
  \@startsection{paragraph}{4}%
  {\z@}{3.25ex \@plus 1ex \@minus .2ex}{1.5ex plus 0.2ex}%
  {\normalfont\normalsize\bfseries\sffamily}%
}

\makeatother


%-----------------------------------
% Zeilenabstand 1,5-zeilig
%-----------------------------------
\usepackage{setspace}
\onehalfspacing

%-----------------------------------
% Absätze durch eine neue Zeile
%-----------------------------------
\setlength{\parindent}{0mm}
\setlength{\parskip}{0.8em plus 0.5em minus 0.3em}

\sloppy					%Abstände variieren
\pagestyle{headings}

%-----------------------------------
% Abkürzungsverzeichnis
%-----------------------------------
\usepackage[intoc]{nomencl}
\renewcommand{\nomname}{Abkürzungsverzeichnis}
\setlength{\nomlabelwidth}{.20\textwidth}
\renewcommand{\nomlabel}[1]{#1 \dotfill}
\setlength{\nomitemsep}{-\parsep}
\makenomenclature

%-----------------------------------
% Meta informationen
%-----------------------------------
%-----------------------------------
% Meta Informationen zur Arbeit
%-----------------------------------

% Autor
\newcommand{\myAutor}{Fiete Ostkamp, Verena Rakers und Artur Gergert}

% Adresse
\newcommand{\myAdresse}{Breul 23 \\ \> \> \> 48147 Münster}

% Titel der Arbeit
\newcommand{\myTitel}{Object Detection auf Basis von Satellitenbildern}
\newcommand{\mySubTitel}{}
% Betreuer
\newcommand{\myBetreuer}{Dipl. Ing. Mustafa Er}

% Lehrveranstaltung
\newcommand{\myLehrveranstaltung}{Big Data Analyseprojekt}

% Matrikelnummer
\newcommand{\myMatrikelNr}{557851, 536491 , 562394}

% Ort
\newcommand{\myOrt}{Münster}

% Datum der Abgabe
\newcommand{\myAbgabeDatum}{\today}

% Semesterzahl
\newcommand{\mySemesterZahl}{2}

% Name der Hochschule
\newcommand{\myHochschulName}{FOM Hochschule für Oekonomie \& Management}

% Standort der Hochschule
\newcommand{\myHochschulStandort}{Münster}

% Studiengang
\newcommand{\myStudiengang}{Big Data \& Business Analytics}

% Art der Arbeit
\newcommand{\myThesisArt}{Hausarbeit}

% Zu erlangender akademische Grad
\newcommand{\myAkademischerGrad}{Master of Science (M. Sc.)}

% Firma
\newcommand{\myFirma}{}


\ifdefined\FOMEN
%Englisch
\entrue
\usepackage[english]{babel}
\else
%Deutsch
\detrue
\usepackage[ngerman]{babel}
\fi

%-----------------------------------
% Kopfbereich / Header definieren
%-----------------------------------
\pagestyle{fancy}
\fancyhf{}
\fancyhead[R]{\thepage}								% Seitenzahl oben, rechts
%\fancyhead[L]{\leftmark}							% kein Footer vorhanden
\renewcommand{\headrulewidth}{0.4pt}


%-----------------------------------
% Start the document here:
%-----------------------------------
\begin{document}

\pagenumbering{Roman}								% Seitennumerierung auf römisch umstellen
\renewcommand{\refname}{Literaturverzeichnis}		% "Literatur" in
%"Literaturverzeichnis" umbenennen
\newcolumntype{C}{>{\centering\arraybackslash}X}	% Neuer Tabellen-Spalten-Typ:
%Zentriert und umbrechbar

%-----------------------------------
% Titlepage
%-----------------------------------
\begin{titlepage}
	\newgeometry{left=2cm, right=2cm, top=2cm, bottom=2cm}
	\begin{center}
		\textbf{\myHochschulName}\\
		\textbf{\myHochschulStandort}\\
		\vspace{1.5cm}
			\includegraphics[width=3cm]{abbildungen/fomLogo.jpg} \\
		\vspace{1.5cm}
		Berufsbegleitender Studiengang\\
		\myStudiengang, \mySemesterZahl. Semester\\
		\vspace{2cm}
		\textbf{\myThesisArt}\\
		\textbf{zur Erlangung des Grades eines}\\
		\textbf{\myAkademischerGrad}\\
		% Oder für Hausarbeiten:		
		%\textbf{im Rahmen der Lehrveranstaltung}\\
		%\textbf{\myLehrveranstaltung}\\
		\vspace{2cm}
		über das Thema\\
		\Huge{\myTitel}\\
		\vspace{0.2cm}
	\end{center}
	\normalsize
	\vfill
	\begin{tabbing}
		Links \= Mitte \= Rechts\kill
		Betreuer: \> \> \myBetreuer\\
		\> \> \\
		
		Autor: \> \> \myAutor\\
		\> \>  Matrikelnr.: \myMatrikelNr\\
		\> \> \myAdresse\\
		\> \> \\
		Abgabe: \> \> \myAbgabeDatum
	\end{tabbing}
\end{titlepage}

%-------Ende Titelseite-------------

%-----------------------------------
% Sperrvermerk
%-----------------------------------
%\newpage
\thispagestyle{empty}

%-----------------------------------
% Sperrvermerk
%-----------------------------------
\section*{Sperrvermerk}
Die vorliegende Abschlussarbeit mit dem Titel \enquote{\myTitel} enthält unternehmensinterne Daten der Firma \myFirma . Daher ist sie nur zur Vorlage bei der FOM sowie den Begutachtern der Arbeit bestimmt. Für die Öffentlichkeit und dritte Personen darf sie nicht zugänglich sein.

\par\medskip
\par\medskip

\_\_\_\_\_\_\_\_\_\_\_\_\_\_\_\_\_\_\_\_\_\_\_\_ \hspace{1.5cm} \_\_\_\_\_\_\_\_\_\_\_\_\_\_\_\_\_\_\_\_\_\_\_\_ \\
(Ort, Datum)\hspace{4.5cm}
(Eigenhändige Unterschrift)

\newpage

%-----------------------------------
% Inhaltsverzeichnis
%-----------------------------------
\setcounter{page}{1}
\tableofcontents
\newpage

%-----------------------------------
% Abkürzungsverzeichnis
%-----------------------------------
\printnomenclature
\newpage
%-----------------------------------
% Abbildungsverzeichnis
%-----------------------------------
\listoffigures
\newpage
%-----------------------------------
% Tabellenverzeichnis
%-----------------------------------
\listoftables
\newpage
%-----------------------------------
% Seitennummerierung auf arabisch und ab 1 beginnend umstellen
%-----------------------------------
\pagenumbering{arabic}
\setcounter{page}{1}
%-----------------------------------
% Kapitel / Inhalte
%-----------------------------------
\section{Einleitung}
Dies soll eine \LaTeX{} -Vorlage für den persönlichen Gebrauch werden. Sie hat weder einen Anspruch auf Richtigkeit, noch auf Vollständigkeit. Die Quellen liegen auf Github zur allgemeinen Verwendung. Verbesserungen sind jederzeit willkommen. 

\subsection{Zielsetzung}
Kleiner Reminder für mich in Bezug auf die Dinge, die wir bei der Thesis beachten sollten und \LaTeX{}-Vorlage für die Thesis.

\subsection{Aufbau der Arbeit}
Kapitel 2 enthält die Inhalte des Thesis-Days und alles, was zum inhaltlichen erstellen der Thesis relevant sein könnte. Kapitel 3 wichtige Anmerkungen zu \LaTeX{}, wobei die wirklich wichtigen Dinge im Quelltext dieses Dokumentes stehen. 

\begin{figure}[H]
\begin{center}
\includegraphics[width=0.9\textwidth]{verzeichnisStruktur}
\caption{Verzeichnisstruktur der \LaTeX{}-Datein}
\end{center}
\end{figure}

\newpage
\section{Grundlagen} \label{sec:grundlagen}
Siehe auch Wissenschaftliches Arbeiten~\footcite[\vglf][S. 1]{Balzert.2008}. %ohne textcommands
Damit sollten alle wichtigen Informationen abgedeckt sein ;-)~\footcite[\vglf][\pagef 1]{Balzert.2008} %mit textcommands
Hier gibt es noch ein Beispiel für ein direktes Zitat\footcite[][\pagef 1]{Balzert.2008} %mit textcommands

\subsection{Grundlagen von [Semantic Segmentation]}
\subsection{Grundlagen von Satellitenbildern}
\subsubsection{3 vs 16 Band}
\subsubsection{GeoTiff}
\subsection{CNN's}
\subsection{Jaccard Score}
\subsection{Vorgehen/Methodik}
\subsubsection{IBM Prozess|CRISP mit HOT OSM Business Case}
\newpage
\section{Praxis} \label{latexDetails}
\subsection{Erstellung eines domänenspezifischen NER Modells}
Wie im Kapitel \ref{NER} erläutert, lässt sich zwischen allgemeinen und domänenspezifischen \ac{NER}-Modellen unterscheiden.\footcite[vgl.][S.47]{nouvel2016} Domänenspezifische Modelle sind dabei solche, die für einen dedizierten Themenbereich erstellt wurden und daher in diesem Kontext besonders gute Ergebnisse erzielen.
Es gibt bereits spezialisierte Modelle im Bereich der Medizin die mit einem medizinischen Korpus trainiert wurden. Dazu zählen zum Beispiel BioBERT, ScispaCy oder Y\footcite[S.12]{li2020}. Im folgenden beschreiben wir, wie wir ein eigenes Modell für das Themengebiet der Urtikaria-Forschung erstellt haben.
\subsubsection{Labeling von Trainingsdaten}
Das Annotieren von Trainingsdaten hat einen zentralen Anteil bei der Erstellung eines eigenen Modells. Akkurate Daten sind essentiell für die Genauigkeit des resultierenden Modells. Neben der Menge der zum Training zur Verfügung stehenden Daten ist es unterlässlich, dass diese widerspruchsfrei sind.

\subsubsection{Auswahl der Labels}
Die Auswahl der Labels muss für den Anwendungsfall angemessen sein. Je nachdem was die Zielstellung des Projektes ist, können unterschieliche Labels erforderlich sein, um die notwendigen Zusammenhänge abzubilden.
In der vorliegenden Arbeit wurden X Entitäten ausgewählt. Diese sind Disease, Treatment, .... Die hier getroffene Auswahl basiert zum einen auf in der einschlägigen Literatur gewählten Labels und zum anderem auf den im Rahmen des CAPTUM Projektes umrissenen Entitäten.

\subsubsection{Annotierungsaufgaben}

\subsubsection{Richtlinien zur Annotation}
\footcite[vgl.][]{neves2014}

\subsubsection{Verwendete Software}
Für die Annotation der Entitäten wurde Doccano verwendet.\footcite[vgl.][S.]{hirokinakayama2021} Dabei handelt es sich um ein open source Tool zur Text Klassifikation, Sequenzlabeling und Sequenz zu Sequenz Aufgaben.
Zur Unterstützung der Annotierung und um den Aufwand zu reduzieren, wurde auf das Auto Labeling Feature der Software zurückgegriffen. Dabei kann eine Backendanwendung konfiguriert werden, die auf Basis eines trainierten Modells Vorschläge zur Annotation unterbreitet.
Für diese Funktionalität wurden von uns zunächst die ersten 100 Textabschnitte ohne Backend annotiert und die gewonnenen Daten dann für das initiale Modell verwendet.
\footcite[vgl.][]{neves2014a}

\subsubsection{Analyse zum Inter-Annotator Agreement \ac{IAA}}\label{sec:IAA}
Das \acf*{IAA} ist ein Maß der Übereinstimmung von Annotationen die von mehreren Personen getätigt wurden. Von dem Score lassen sich allgemein Rückschlüsse ziehen, wie zuverlässig der Annotierungsprozess ablief. \footcite[vgl.][S.298]{ide2017} Der Grundgedanke dabei ist, dass ein hoher Score für die Reproduzierbarkeit der Ergebnisse spricht und Ausdruck ist von der Klarheit der Richtlinien.

% https://towardsdatascience.com/inter-annotator-agreement-2f46c6d37bf3
\begin{itemize}
    \item Cohens K\footcite[vgl.][S.]{cohen1960}
    \item Fleiss K\footcite[vgl.][S.]{fleiss1971}
\end{itemize}

\subsubsection{Labeling von Trainingsdaten}
Das Labeling der Daten wurde von zwei Personen ohne medizinischen Hintergrund ausgeführt. Um dennoch eine möglichst hohe Qualität der Annotationen zu erhalten, haben die teilnehmenden Personen bei Unklarheit im Internet recherchiert. Wie im Abschnitt \ref{sec:IAA} beschrieben ist, wurde dadurch dennoch eine verhältnismäßig gute Genauigkeit erzielt.
Insgesamt wurden dadurch X Textabschnitte annotiert, die zufällig aus dem Gesamtcorpus von 465 einzigartigen Texten ausgewählt wurden. Dies entspricht einem Anteil von circa 0.01\%, relativ gesehen zu der Gesamtzahl von 57764 Abschnitten.

% Das annotieren von Trainingsdaten hat einen zentralen Anteil bei der Erstellung eines eigenen Modells. Akkurate Daten sind essentiell für die Genauigkeit des resultierenden Modells. Neben der Menge der zum Training zur Verfügung stehenden Daten ist es außerdem unerlässlich, dass diese widerspruchsfrei sind.

\begin{itemize}
    \item Notwendige Menge an Annotationen
    \item Aufstellung der Label (Ontologie?)
    \item Richtlininien zur Annotation
    \item Auswahl der Software (Doccano, Spacy)
\end{itemize}
\subsubsection{Training des Modells}
Für das Training des Modells werden die annotierten Daten aus den vorherigen Schritten verwendet.
Für das Training des Modells wurde die \acl*{NLP} Bibliothek spaCy\footcite[]{spacy2} verwendet. Diese wurde im Hinblick auf ihre umfangreichen Funktionalitäten und benutzerfreundlichkeit ausgewählt.

Das Training erfolgt über den \textit{spacy train} Befehl und wird über ein config file konfiguriert. Die Parameter für das Training des Modells orientieren sich im Wesentlichen an den voreingestellten Werten.\footcite[vgl.]{ostkamp2021} Die Library kommt mit einer Standardkonfig, die für diesen Anwendungsfall ohne Änderungen übernommen wurde.

Für einige \ac*{nlp} tasks wie die \textit{text classification} können unterschiedliche Architekturen gewählt werden, für die \acl*{NER} ist nur der \textit{Transition Based Parser} verfügbar.\footcite{zotero-182}

\begin{itemize}
    \item Warum SpaCy
    \item Modellarchitektur ()
    %https://spacy.io/api/architectures#TransitionBasedParser
    \item Trainingsparameter
\end{itemize}


Bei der zugrundeliegenden Architektur handelt es sich um einen sogenannten Transition Based Parser.\footcite{zotero-182} Diesem liegt das Konzept von Übergängen zwischen Wörtern zugrunde. Auf Basis der Annahme, dass der \acl*{POS} tag eines Wortes abhängig ist von bisherigen Wörtern (dem sogenannten \textit{state}) eines Satzes, lassen sich Wahrscheinlichkeiten für Übergänge vom letzten hin zum nächsten Tag aufstellen.\footcite{honnibal2013a} Diese werden in eine sogenannte \textit{transition matrix} überführt, die die Wahrscheinlichkeiten enthält für die Nachfolge eines Tags auf einen anderen.
\begin{itemize}
    \item Spacy CLI
    \item Training mit default settings
\end{itemize}
Das annotieren von Trainingsdaten hat einen zentralen Anteil bei der Erstellung eines eigenen Modells. Akkurate Daten sind essentiell für die Genauigkeit des resultierenden Modells. Neben der Menge der zum Training zur Verfügung stehenden Daten ist es unterlässlich, dass diese widerspruchsfrei sind.

\subsubsection{Evaluierung des Modells}
\footcite[]{tsai2006}

\subsection{Markeranalyse}
\subsection{Überführung in Tabellenstruktur}
\subsection{Markerkorrelationen}
\subsection{Aufstellung des Gradingsystems}

\newpage
\section{Schlussbetrachtung}
(Umfang ca. 0,5 - 1 Seite)

\subsection{Zusammenfassung}

\subsection{Fazit}

\subsection{Kritische Reflexion}

\subsection{Ausblick}


%-----------------------------------
% Literaturverzeichnis
%-----------------------------------
\newpage
%\addcontentsline{toc}{section}{Literatur}

\pagenumbering{Roman} %Zähler wieder römisch ausgeben
\setcounter{page}{4}  %Zähler manuell hochsetzen

\printbibliography

% Alternative Darstellung:
% Literaturverzeichnis nach Typ (@book, @arcticle ...) sortiert.
% Dazu die Zeile (\printbibliography) auskommentieren und folgenden code verwenden:

%\printbibheading
%\printbibliography[type=article,heading=subbibliography,title={Artikel}]
%\printbibliography[type=book,heading=subbibliography,title={Bücher}]
%\printbibliography[type=online,heading=subbibliography,title={Webseiten}]

\newpage
\pagenumbering{gobble} % Keine Seitenzahlen mehr

%-----------------------------------
% Ehrenwörtliche Erklärung
%-----------------------------------
\section*{%
	\langde{Ehrenwörtliche Erklärung}
	\langen{Declaration in lieu of oath}}
\langde{Hiermit versichere ich, dass die vorliegende Arbeit von mir selbstständig und ohne unerlaubte Hilfe angefertigt worden ist, insbesondere dass ich alle Stellen, die wörtlich oder annähernd wörtlich aus Veröffentlichungen entnommen sind, durch Zitate als solche gekennzeichnet habe. Ich versichere auch, dass die von mir eingereichte schriftliche Version mit der digitalen Version übereinstimmt. Weiterhin erkläre ich, dass die Arbeit in gleicher oder ähnlicher Form noch keiner Prüfungsbehörde/Prüfungsstelle vorgelegen hat. Ich erkläre mich damit \textcolor{red}{einverstanden/nicht einverstanden}, dass die Arbeit der Öffentlichkeit zugänglich gemacht wird. Ich erkläre mich damit einverstanden, dass die Digitalversion dieser Arbeit zwecks Plagiatsprüfung auf die Server externer Anbieter hochgeladen werden darf. Die Plagiatsprüfung stellt keine Zurverfügungstellung für die Öffentlichkeit dar.}
\langen{I hereby declare that I produced the submitted paper with no assistance from any other party and without the use of any unauthorized aids and, in particular, that I have marked as quotations all passages which are reproduced verbatim or near-verbatim from publications. Also, I declare that the submitted print version of this thesis is identical with its digital version. Further, I declare that this thesis has never been submitted before to any examination board in either its present form or in any other similar version. I herewith \textcolor{red}{agree/disagree} that this thesis may be published. I herewith consent that this thesis may be uploaded to the server of external contractors for the purpose of submitting it to the contractors’ plagiarism detection systems. Uploading this thesis for the purpose of submitting it to plagiarism detection systems is not a form of publication.}


\par\medskip
\par\medskip

\vspace{5cm}

\begin{table}[H]
	\centering
	\begin{tabular*}{\textwidth}{c @{\extracolsep{\fill}} ccccc}
		\myOrt, \the\day.\the\month.\the\year
		&
		% Hinterlege deine eingescannte Unterschrift im Verzeichnis /abbildungen und nenne sie unterschrift.png
		% Bilder mit transparentem Hintergrund können teils zu Problemen führen
		\includegraphics[width=0.35\textwidth]{unterschrift}\vspace*{-0.35cm}
		\\
		\rule[0.5ex]{12em}{0.55pt} & \rule[0.5ex]{12em}{0.55pt} \\
		\langde{(Ort, Datum)}\langen{(Location, Date)} & \langde{(Eigenhändige Unterschrift)}\langen{(handwritten signature)}
		\\
	\end{tabular*} \\
\end{table}

\end{document}
