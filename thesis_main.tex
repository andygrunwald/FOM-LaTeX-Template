%-----------------------------------
% Define document and include general packages
%-----------------------------------
% Tabellen- und Abbildungsverzeichnis stehen normalerweise nicht im
% Inhaltsverzeichnis. Gleiches gilt für das Abkürzungsverzeichnis (siehe unten).
% Manche Dozenten bemängeln das. Die Optionen 'listof=totoc,bibliography=totoc'
% geben das Tabellen- und Abbildungsverzeichnis im Inhaltsverzeichnis (toc=Table
% of Content) aus.
% Da es aber verschiedene Regelungen je nach Dozent geben kann, werden hier
% beide Varianten dargestellt.
\documentclass[12pt,oneside,titlepage,listof=totoc,bibliography=totoc]{scrartcl}
%\documentclass[12pt,oneside,titlepage]{scrartcl}

%-----------------------------------
% Dokumentensprache
%-----------------------------------
%\def\FOMEN{}% Auskommentieren um die Dokumentensprache auf englisch zu ändern
\newif\ifde
\newif\ifen

%-----------------------------------
% Meta informationen
%-----------------------------------
%-----------------------------------
% Meta Informationen zur Arbeit
%-----------------------------------

% Autor
\newcommand{\myAutor}{Fiete Ostkamp, Verena Rakers und Artur Gergert}

% Adresse
\newcommand{\myAdresse}{Breul 23 \\ \> \> \> 48147 Münster}

% Titel der Arbeit
\newcommand{\myTitel}{Object Detection auf Basis von Satellitenbildern}
\newcommand{\mySubTitel}{}
% Betreuer
\newcommand{\myBetreuer}{Dipl. Ing. Mustafa Er}

% Lehrveranstaltung
\newcommand{\myLehrveranstaltung}{Big Data Analyseprojekt}

% Matrikelnummer
\newcommand{\myMatrikelNr}{557851, 536491 , 562394}

% Ort
\newcommand{\myOrt}{Münster}

% Datum der Abgabe
\newcommand{\myAbgabeDatum}{\today}

% Semesterzahl
\newcommand{\mySemesterZahl}{2}

% Name der Hochschule
\newcommand{\myHochschulName}{FOM Hochschule für Oekonomie \& Management}

% Standort der Hochschule
\newcommand{\myHochschulStandort}{Münster}

% Studiengang
\newcommand{\myStudiengang}{Big Data \& Business Analytics}

% Art der Arbeit
\newcommand{\myThesisArt}{Hausarbeit}

% Zu erlangender akademische Grad
\newcommand{\myAkademischerGrad}{Master of Science (M. Sc.)}

% Firma
\newcommand{\myFirma}{}


\ifdefined\FOMEN
%Englisch
\entrue
\usepackage[english]{babel}
\else
%Deutsch
\detrue
\usepackage[ngerman]{babel}
\fi

\newcommand{\langde}[1]{%
   \ifde\selectlanguage{ngerman}#1\fi}
\newcommand{\langen}[1]{%
   \ifen\selectlanguage{english}#1\fi}
\usepackage[utf8]{luainputenc}
\langde{\usepackage[babel,german=quotes]{csquotes}}
\langen{\usepackage[babel,english=british]{csquotes}}
\usepackage[T1]{fontenc}
\usepackage{fancyhdr}
\usepackage{fancybox}
\usepackage[a4paper, left=4cm, right=2cm, top=4cm, bottom=2cm]{geometry}
\usepackage{graphicx}
\usepackage{colortbl}
\usepackage[capposition=top]{floatrow}
\usepackage{array}
\usepackage{float}      %Positionierung von Abb. und Tabellen mit [H] erzwingen
\usepackage{footnote}
% Darstellung der Beschriftung von Tabellen und Abbildungen (Leitfaden S. 44)
% singlelinecheck=false: macht die Caption linksbündig (statt zentriert)
% labelfont auf fett: (Tabelle x.y:, Abbildung: x.y)
% font auf fett: eigentliche Bezeichnung der Abbildung oder Tabelle
% Fettschrift laut Leitfaden 2018 S. 45
\usepackage[singlelinecheck=false, labelfont=bf, font=bf]{caption}
\usepackage{caption}
\usepackage{enumitem}
\usepackage{amssymb}
\usepackage{mathptmx}
%\usepackage{minted} %Kann für schöneres Syntax Highlighting genutzt werden. ACHTUNG: Python muss installiert sein.
\usepackage[scaled=0.9]{helvet} % Behebt, zusammen mit Package courier, pixelige Überschriften. Ist, zusammen mit mathptx, dem times-Package vorzuziehen. Details: https://latex-kurs.de/fragen/schriftarten/Times_New_Roman.html
\usepackage{courier}
\usepackage{amsmath}
\usepackage[table]{xcolor}
\usepackage{marvosym}			% Verwendung von Symbolen, z.B. perfektes Eurozeichen

\renewcommand\familydefault{\sfdefault}
\usepackage{ragged2e}

% Mehrere Fussnoten nacheinander mit Komma separiert
\usepackage[hang,multiple]{footmisc}
\setlength{\footnotemargin}{1em}

% todo Aufgaben als Kommentare verfassen für verschiedene Editoren
\usepackage{todonotes}

% Verhindert, dass nur eine Zeile auf der nächsten Seite steht
\setlength{\marginparwidth}{2cm}
\usepackage[all]{nowidow}

%-----------------------------------
% Farbdefinitionen
%-----------------------------------
\definecolor{darkblack}{rgb}{0,0,0}
\definecolor{dunkelgrau}{rgb}{0.8,0.8,0.8}
\definecolor{hellgrau}{rgb}{0.0,0.7,0.99}
\definecolor{mauve}{rgb}{0.58,0,0.82}
\definecolor{dkgreen}{rgb}{0,0.6,0}

%-----------------------------------
% Pakete für Tabellen
%-----------------------------------
\usepackage{epstopdf}
\usepackage{nicefrac} % Brüche
\usepackage{multirow}
\usepackage{rotating} % vertikal schreiben
\usepackage{mdwlist}
\usepackage{tabularx}% für Breitenangabe

%-----------------------------------
% sauber formatierter Quelltext
%-----------------------------------
\usepackage{listings}
% JavaScript als Sprache definieren:
\lstdefinelanguage{JavaScript}{
	keywords={break, super, case, extends, switch, catch, finally, for, const, function, try, continue, if, typeof, debugger, var, default, in, void, delete, instanceof, while, do, new, with, else, return, yield, enum, let, await},
	keywordstyle=\color{blue}\bfseries,
	ndkeywords={class, export, boolean, throw, implements, import, this, interface, package, private, protected, public, static},
	ndkeywordstyle=\color{darkgray}\bfseries,
	identifierstyle=\color{black},
	sensitive=false,
	comment=[l]{//},
	morecomment=[s]{/*}{*/},
	commentstyle=\color{purple}\ttfamily,
	stringstyle=\color{red}\ttfamily,
	morestring=[b]',
	morestring=[b]"
}

\lstset{
	%language=JavaScript,
	numbers=left,
	numberstyle=\tiny,
	numbersep=5pt,
	breaklines=true,
	showstringspaces=false,
	frame=l ,
	xleftmargin=5pt,
	xrightmargin=5pt,
	basicstyle=\ttfamily\scriptsize,
	stepnumber=1,
	keywordstyle=\color{blue},          % keyword style
  	commentstyle=\color{dkgreen},       % comment style
  	stringstyle=\color{mauve}         % string literal style
}

%-----------------------------------
%Literaturverzeichnis Einstellungen
%-----------------------------------

% Biblatex

\usepackage{url}
\urlstyle{same}

%%%% Neuer Leitfaden (2018)
\usepackage[
backend=biber,
style=ext-authoryear-ibid, % Auskommentieren und nächste Zeile einkommentieren, falls "Ebd." (ebenda) nicht für sich-wiederholende Fussnoten genutzt werden soll.
%style=ext-authoryear,
maxcitenames=3,	% mindestens 3 Namen ausgeben bevor et. al. kommt
maxbibnames=999,
mergedate=false,
date=iso,
seconds=true, %werden nicht verwendet, so werden aber Warnungen unterdrückt.
urldate=iso,
innamebeforetitle,
dashed=false,
autocite=footnote,
doi=false,
useprefix=true, % 'von' im Namen beachten (beim Anzeigen)
mincrossrefs = 1
]{biblatex}%iso dateformat für YYYY-MM-DD

%weitere Anpassungen für BibLaTex
\usepackage{xpatch} 

\setlength\bibhang{1cm} 

%%% Weitere Optionen 
%\boolitem[false]{citexref} %Wenn incollection, inbook, inproceedings genutzt wird nicht den zugehörigen parent auch in Literaturverzeichnis aufnehmen

%Aufräumen die Felder werden laut Leitfaden nicht benötigt.
\AtEveryBibitem{%
\ifentrytype{book}{
    \clearfield{issn}%
    \clearfield{doi}%
    \clearfield{isbn}%
    \clearfield{url}
    \clearfield{eprint}
}{}
\ifentrytype{collection}{
  \clearfield{issn}%
  \clearfield{doi}%
  \clearfield{isbn}%
  \clearfield{url}
  \clearfield{eprint}
}{}
\ifentrytype{incollection}{
  \clearfield{issn}%
  \clearfield{doi}%
  \clearfield{isbn}%
  \clearfield{url}
  \clearfield{eprint}
}{}
\ifentrytype{article}{
  \clearfield{issn}%
  \clearfield{doi}%
  \clearfield{isbn}%
  \clearfield{url}
  \clearfield{eprint}
}{}
}

\renewcommand*{\finentrypunct}{}%Kein Punkt am ende des Literaturverzeichnisses

\renewcommand*{\newunitpunct}{\addcomma\space} 
\DeclareDelimFormat[bib,biblist]{nametitledelim}{\addcolon\space} 
\DeclareDelimFormat{titleyeardelim}{\newunitpunct} 
%Namen kursiv schreiben
\renewcommand*{\mkbibnamefamily}{\mkbibemph} 
\renewcommand*{\mkbibnamegiven}{\mkbibemph} 
\renewcommand*{\mkbibnamesuffix}{\mkbibemph} 
\renewcommand*{\mkbibnameprefix}{\mkbibemph} 

% Die Trennung mehrerer Autorennamen erfolgt durch Kommata.
% siehe Beispiele im Leitfaden S. 16
% Die folgende Zeile würde mit Semikolon trennen
%\DeclareDelimFormat{multinamedelim}{\addsemicolon\addspace}

%Delimiter für mehrere und letzten Namen gleich setzen
\DeclareDelimAlias{finalnamedelim}{multinamedelim} 

\DeclareNameAlias{default}{family-given} 
\DeclareNameAlias{sortname}{default}  %Nach Namen sortieren


\DeclareFieldFormat{editortype}{\mkbibparens{#1}}
\DeclareDelimFormat{editortypedelim}{\addspace} 
\DeclareFieldFormat{translatortype}{\mkbibparens{#1}} 
\DeclareDelimFormat{translatortypedelim}{\addspace} 
\DeclareDelimFormat[bib,biblist]{innametitledelim}{\addcomma\space} 

\DeclareFieldFormat*{citetitle}{#1} 
\DeclareFieldFormat*{title}{#1} 
\DeclareFieldFormat*{booktitle}{#1} 
\DeclareFieldFormat*{journaltitle}{#1} 

\xpatchbibdriver{online} 
  {\usebibmacro{organization+location+date}\newunit\newblock} 
  {} 
  {}{} 

\DeclareFieldFormat[online]{date}{\mkbibparens{#1}} 
\DeclareFieldFormat{urltime}{#1\addspace Uhr}
\DeclareFieldFormat{urldate}{%urltime zu urldate hinzufügen
  [Zugriff\addcolon\addspace
  #1
  \printfield{urltime}]
}
\DeclareFieldFormat[online]{url}{\mkbibacro{URL}\addcolon\space <\url{#1}>}
\renewbibmacro*{url+urldate}{% 
  \usebibmacro{url}% 
  \ifentrytype{online} 
    {\setunit*{\addspace}% 
     \iffieldundef{year}
       {\printtext[date]{keine Datumsangabe}} 
       {\usebibmacro{date}}}% 
    {}% 
  \setunit*{\addspace}% 
  \usebibmacro{urldate}
  } 


\renewbibmacro*{date+extradate}{% 
  \printtext[parens]{% 
    \printfield{usera}% 
    \setunit{\printdelim{titleyeardelim}}% 
    \printlabeldateextra}} 

\DefineBibliographyStrings{german}{ 
  nodate    = {{}o.\adddot J\adddot}, 
  andothers = {et\addabbrvspace al\adddot} 
} 

\DeclareSourcemap{ 
  \maps[datatype=bibtex]{ 
    \map{ 
      \step[notfield=translator, final] 
      \step[notfield=editor, final] 
      \step[fieldset=author,
      fieldvalue={{{o\noexpand\adddot\addspace V\noexpand\adddot}}}] } 
    \map{ 
      \pernottype{online} 
      \step[fieldset=location,
      fieldvalue={o\noexpand\adddot\addspace O\noexpand\adddot}] } 
  } 
} 

\renewbibmacro*{cite}{% 
  \iffieldundef{shorthand} 
    {\ifthenelse{\ifnameundef{labelname}\OR\iffieldundef{labelyear}} 
       {\usebibmacro{cite:label}% 
        \setunit{\printdelim{nonametitledelim}}} 
       {\printnames{labelname}% 
        \setunit{\printdelim{nametitledelim}}}% 
     \printfield{usera}% 
     \setunit{\printdelim{titleyeardelim}}% 
     \usebibmacro{cite:labeldate+extradate}} 
    {\usebibmacro{cite:shorthand}}} 

    \renewcommand*{\jourvoldelim}{\addcomma\addspace}% Trennung zwischen journalname und Volume. Sonst Space; Laut Leitfaden richtig
    \hypersetup{hidelinks} %sonst sind Fußnoten grün. Dadurch werden Links allerdings nicht mehr farbig dargestellt

\renewbibmacro*{journal+issuetitle}{%
  \usebibmacro{journal}%
  \setunit*{\jourvoldelim}%
  \iffieldundef{series}
    {}
    {\setunit*{\jourserdelim}%
     \printfield{series}%
     \setunit{\servoldelim}}%
  \iffieldundef{volume}
    {}
    {\printfield{volume}}
  \iffieldundef{labelyear}
  {}
  {
  (\thefield{year}) %Ansonsten wird wenn kein Volume angegeben ist ein Komma vorangestellt
  }
  \setunit*{\addcomma\addspace Nr\adddot\addcolon\addspace}
  \printfield{number}
  \iffieldundef{eid}
  {}
  {\printfield{eid}}
}

% Postnote ist der Text in der zweiten eckigen Klammer bei einem Zitat
% wenn es keinen solchen Eintrag gibt, dann auch nicht ausgeben, z.B. 'o. S.'
% Wenn man das will, kann man das 'o. S.' ja explizit angeben. Andernfalls steht
% sonst auch bei Webseiten 'o. S.' da, was laut Leitfaden nicht ok ist.
\renewbibmacro*{postnote}{% 
  \setunit{\postnotedelim}% 
  \iffieldundef{postnote} 
    {} %{\printtext{o\adddot\addspace S\adddot}} 
    {\printfield{postnote}}} 
    
% Abstand bei Änderung Anfangsbuchstabe ca. 1.5 Zeilen
\setlength{\bibinitsep}{0.75cm}

% nur in den Zitaten/Fussnoten den Vornamen abkürzen (nicht im
% Literaturverzeichnis)

\DeclareDelimFormat{nonameyeardelim}{\addcomma\space} 
\DeclareDelimFormat{nameyeardelim}{\addcomma\space} 

\renewbibmacro*{cite}{%
  \iffieldundef{shorthand}
    {\ifthenelse{\ifnameundef{labelname}\OR\iffieldundef{labelyear}}
       {\usebibmacro{cite:label}%
        \setunit{\printdelim{nonameyeardelim}}}
      {\toggletrue{abx@bool@giveninits}
        \printnames[family-given]{labelname}%
        \setunit{\printdelim{nameyeardelim}}}%
      \printfield{usera}%
      \setunit{\printdelim{titleyeardelim}}% 
     \usebibmacro{cite:labeldate+extradate}}
   {\usebibmacro{cite:shorthand}}}


%%%%% Alter Leitfaden. Ggf. Einkommentieren und Bereich hierüber auskommentieren
%\usepackage[
%backend=biber,
%style=numeric,
%citestyle=authoryear,
%url=false,
%isbn=false,
%notetype=footonly,
%hyperref=false,
%sortlocale=de]{biblatex}

%weitere Anpassungen für BibLaTex
%% Optionen für Biblatex
% Laut Leitfaden:
% - ist es besser den Vornamen in Kurzform anzugeben
% - Bücher ohne ISBN anzugeben
\ExecuteBibliographyOptions{%
giveninits=true,
isbn=false,
url=true,
doi=false,
eprint=false,
maxbibnames=7, % Alle Autoren (kein et al.)
maxcitenames=2, % et al. ab dem 3. Autor
backref=false, % Rückverweise auf Zitatseiten
bibencoding=utf8, % wenn .bib in utf8, sonst ascii
bibwarn=true, % Warnung bei fehlerhafter bib-Datei
}%

% et al. an Stelle von u.a.
\DefineBibliographyStrings{ngerman}{
   andothers = {{et\,al\adddot}},
}

% Klammern um das Jahr in der Fußnote
\renewbibmacro*{cite:labelyear+extrayear}{%
  \iffieldundef{labelyear}
    {}
    {\printtext[bibhyperref]{%
       \mkbibparens{%
         \printfield{labelyear}%
         \printfield{extrayear}}}}}

\DeclareNameFormat{last-first}{%
  \iffirstinits
    {\usebibmacro{name:family-given}
        {\namepartfamily}
        {\namepartgiveni}
        {\namepartprefix}
        {\namepartsuffix}
    }
    {\usebibmacro{name:family-given}
        {\namepartfamily}
        {\namepartgiven}
        {\namepartprefix}
        {\namepartsuffix}
    }%
  \usebibmacro{name:andothers}}

% Alternative Notation der Fußnoten
% Zeigt sowohl den Nachnamen als auch den Vornamen an
% Beispiel: \fullfootcite[Vgl. ][Seite 5]{Tanenbaum.2003}
\DeclareCiteCommand{\fullfootcite}[\mkbibfootnote]
  {\usebibmacro{prenote}}
  {\usebibmacro{citeindex}%
    \printnames[sortname][1-1]{author}%
    \addspace (\printfield{year})}
  {\addsemicolon\space}
  {\usebibmacro{postnote}}

%Autoren (Nachname, Vorname)
\DeclareNameAlias{default}{family-given}

% Laut Leitfaden sind:
% - die Titel (egal welche) ohne Anfuehrungszeichen
% - Anpassung von @Inbook elementen
% - Anpassung von Online Quellen
\DeclareFieldFormat[article,book,inbook,inproceedings]{title}{{#1}}
\DeclareFieldFormat[inproceedings]{booktitle}{{#1}}
\DeclareFieldFormat[inbook]{chapter}{{#1}}

% Laut Leitfaden...
\DeclareFieldFormat{urldate}{\addcomma\space\bibstring{urlseen}\space#1}
\DefineBibliographyStrings{german}{%
	urlseen = {Abruf am},
}

% Reihenfolge von publisher, year, address verändern

% Definiert @Book Eintrag
% Laut Leitfaden ...
\DeclareBibliographyDriver{book}{%
  \printnames{author}%
  \setunit*{\space (}%
  \printfield{year}\newunit{)}%
  \newunit\addcolon\space
  \printfield{title}%
  \setunit*{,\space}%
  \printlist{location}%
  \setunit*{\space}%
  \printfield{year}%
  \finentry}

% Add @Inbook style
% Laut Leitfaden ...
\DeclareBibliographyDriver{inbook}{%
  \printnames{author}%
  \setunit*{\space (}%
  \printfield{year}\newunit{)}%
  \newunit\addcolon\space
  \printfield{chapter}%
  \setunit*{, in:\space}%
  \printnames{editor}\newunit{\space (Hrsg.)}%
  \setunit*{, \space}%
  \printfield{title}%
  \setunit*{,\space}%
  \printlist{location}%
  \setunit*{\space}%
  \printfield{year}%
  \setunit*{, \space}%
  \printfield{pages}%
  \finentry}

% Definiert @Online Eintrag
% Laut Leitfaden ...
\DeclareBibliographyDriver{online}{%
  \printnames{author}%
  \newunit\newblockpunct
  \printfield{title}%
  \setunit*{,\space}%
  %\newunit\newblock
  \printfield{url}%
  \setunit*{,\space Erscheinungsjahr:\space}%
  \printfield{year}%
  \setunit*{,\space Aufruf am:\space}%
  \printfield{note}%
  \finentry}

% Definiert @Article Eintrag
% Laut Leitfaden ...
\DeclareBibliographyDriver{article}{%
  \printnames{author}%
  \setunit*{\space (}%
  \printfield{year}\newunit{)}%
  \newunit\addcolon\space
  \printfield{title}%
  \setunit*{.\space In:\space}%
  %\newunit\newblock
  \usebibmacro{journal}%
  \setunit*{\space }%
  \printfield{year}%
  \setunit*{, \space}%
  \setunit*{, Nr. \space}%
  \printfield{number}\newunit{, \space}%
  \printfield{pages}%
  \finentry}

% Definiert @Inproceeedings Eintrag
% (booktitle != title) - das ist Grund zum unterschied @Article
% Laut Leitfaden ...
\DeclareBibliographyDriver{inproceedings}{%
  \printnames{author}%
  \setunit*{\space (}%
  \printfield{year}\newunit{)}%
  \newunit\addcolon\space
  \printfield{title}%
  \setunit*{.\space In:\space}%
  %\newunit\newblock
  \printfield{booktitle}%
  \setunit*{\space }%
  \printfield{year}%
  \setunit*{, \space}%
  \setunit*{, Nr. \space}%
  \printfield{number}\newunit{, \space}%
  \printfield{pages}%
  \finentry}


%Doppelpunkt nach dem letzten Autor
\renewcommand*{\labelnamepunct}{\addcolon\addspace }

%Komma an Stelle des Punktes
\renewcommand*{\newunitpunct}{\addcomma\space}

%Autoren durch Semikolon trennen
\newcommand*{\bibmultinamedelim}{\addsemicolon\space}%
\newcommand*{\bibfinalnamedelim}{\addsemicolon\space}%
\AtBeginBibliography{%
  \let\multinamedelim\bibmultinamedelim
  \let\finalnamedelim\bibfinalnamedelim
}

%Titel nicht kursiv anzeigen
\DeclareFieldFormat{title}{#1\isdot}


%%%% Ende Alter Leitfaden

%Bib-Datei einbinden
\addbibresource{literatur/literatur.bib}

% Zeilenabstand im Literaturverzeichnis ist Einzeilig
% siehe Leitfaden S. 14
\AtBeginBibliography{\singlespacing}

%-----------------------------------
% Silbentrennung
%-----------------------------------
\usepackage{hyphsubst}
\HyphSubstIfExists{ngerman-x-latest}{%
\HyphSubstLet{ngerman}{ngerman-x-latest}}{}

%-----------------------------------
% Pfad fuer Abbildungen
%-----------------------------------
\graphicspath{{./}{./abbildungen/}}

%-----------------------------------
% Weitere Ebene einfügen
%-----------------------------------
\usepackage{titletoc}

\makeatletter

% Setze die Tiefe des Inhaltsverzeichnis auf 4 Ebenen
% Damit erscheinen \paragraph-Sektionen auch im Inhaltsverzeichnis
\setcounter{secnumdepth}{4}
\setcounter{tocdepth}{4}

% Fuege Abstand nach unten wie in einer normalen \section hinzu
% Andernfalls haette \paragraph keinen Zeilenumbruch
% Der Zeilenumbruch koennte mit einer leeren \mbox{} ersetzt werden
% Jedoch klebt dann der Text relativ nah an der Ueberschrift
\renewcommand{\paragraph}{%
  \@startsection{paragraph}{4}%
  {\z@}{3.25ex \@plus 1ex \@minus .2ex}{1.5ex plus 0.2ex}%
  {\normalfont\normalsize\bfseries\sffamily}%
}

\makeatother


%-----------------------------------
% Paket für die Nutzung von Anhängen
%-----------------------------------
\usepackage{appendix}

%-----------------------------------
% Zeilenabstand 1,5-zeilig
%-----------------------------------
\usepackage{setspace}
\onehalfspacing

%-----------------------------------
% Absätze durch eine neue Zeile
%-----------------------------------
\setlength{\parindent}{0mm}
\setlength{\parskip}{0.8em plus 0.5em minus 0.3em}

\sloppy					%Abstände variieren
\pagestyle{headings}

%----------------------------------
% Präfix in das Abbildungs- und Tabellenverzeichnis aufnehmen, statt nur der Nummerierung (siehe Issue #206).
%----------------------------------
\KOMAoption{listof}{entryprefix} % Siehe KOMA-Script Doku v3.28 S.153
\BeforeStartingTOC[lof]{\renewcommand*\autodot{:}} % Für den Doppelpunkt hinter Präfix im Abbildungsverzeichnis
\BeforeStartingTOC[lot]{\renewcommand*\autodot{:}} % Für den Doppelpunkt hinter Präfix im Tabellenverzeichnis

%-----------------------------------
% Abkürzungsverzeichnis
%-----------------------------------
\usepackage[printonlyused]{acronym}

%-----------------------------------
% Symbolverzeichnis
%-----------------------------------
% Quelle: https://www.namsu.de/Extra/pakete/Listofsymbols.pdf
\usepackage[final]{listofsymbols}

%-----------------------------------
% Glossar
%-----------------------------------
\usepackage{glossaries}
\glstoctrue %Auskommentieren, damit das Glossar nicht im Inhaltsverzeichnis angezeigt wird.
\makenoidxglossaries
\newglossaryentry{glossar}{name={Glossar},description={In einem Glossar werden Fachbegriffe und Fremdwörter mit ihren Erklärungen gesammelt.}}
\newglossaryentry{glossaries}{name={Glossaries},description={Glossaries ist ein Paket was einen im Rahmen von LaTeX bei der Erstellung eines Glossar unterstützt.}}


%-----------------------------------
% PDF Meta Daten setzen
%-----------------------------------
\usepackage[hyperfootnotes=false]{hyperref} %hyperfootnotes=false deaktiviert die Verlinkung der Fußnote. Ansonsten inkompaibel zum Paket "footmisc"
% Behebt die falsche Darstellung der Lesezeichen in PDF-Dateien, welche eine Übersetzung besitzen
% siehe Issue 149
\makeatletter
\pdfstringdefDisableCommands{\let\selectlanguage\@gobble}
\makeatother

\hypersetup{
    pdfinfo={
        Title={\myTitel},
        Subject={\myStudiengang},
        Author={\myAutor},
        Build=1.1
    }
}

%-----------------------------------
% PlantUML
%-----------------------------------
%\usepackage{plantuml}

%-----------------------------------
% Umlaute in Code korrekt darstellen
% siehe auch: https://en.wikibooks.org/wiki/LaTeX/Source_Code_Listings
%-----------------------------------
\lstset{literate=
	{á}{{\'a}}1 {é}{{\'e}}1 {í}{{\'i}}1 {ó}{{\'o}}1 {ú}{{\'u}}1
	{Á}{{\'A}}1 {É}{{\'E}}1 {Í}{{\'I}}1 {Ó}{{\'O}}1 {Ú}{{\'U}}1
	{à}{{\`a}}1 {è}{{\`e}}1 {ì}{{\`i}}1 {ò}{{\`o}}1 {ù}{{\`u}}1
	{À}{{\`A}}1 {È}{{\'E}}1 {Ì}{{\`I}}1 {Ò}{{\`O}}1 {Ù}{{\`U}}1
	{ä}{{\"a}}1 {ë}{{\"e}}1 {ï}{{\"i}}1 {ö}{{\"o}}1 {ü}{{\"u}}1
	{Ä}{{\"A}}1 {Ë}{{\"E}}1 {Ï}{{\"I}}1 {Ö}{{\"O}}1 {Ü}{{\"U}}1
	{â}{{\^a}}1 {ê}{{\^e}}1 {î}{{\^i}}1 {ô}{{\^o}}1 {û}{{\^u}}1
	{Â}{{\^A}}1 {Ê}{{\^E}}1 {Î}{{\^I}}1 {Ô}{{\^O}}1 {Û}{{\^U}}1
	{œ}{{\oe}}1 {Œ}{{\OE}}1 {æ}{{\ae}}1 {Æ}{{\AE}}1 {ß}{{\ss}}1
	{ű}{{\H{u}}}1 {Ű}{{\H{U}}}1 {ő}{{\H{o}}}1 {Ő}{{\H{O}}}1
	{ç}{{\c c}}1 {Ç}{{\c C}}1 {ø}{{\o}}1 {å}{{\r a}}1 {Å}{{\r A}}1
	{€}{{\EUR}}1 {£}{{\pounds}}1 {„}{{\glqq{}}}1
}

%-----------------------------------
% Kopfbereich / Header definieren
%-----------------------------------
\pagestyle{fancy}
\fancyhf{}
% Seitenzahl oben, mittig, mit Strichen beidseits
% \fancyhead[C]{-\ \thepage\ -}

% Seitenzahl oben, mittig, entsprechend Leitfaden ohne Striche beidseits
\fancyhead[C]{\thepage}
%\fancyhead[L]{\leftmark}							% kein Footer vorhanden
% Waagerechte Linie unterhalb des Kopfbereiches anzeigen. Laut Leitfaden ist
% diese Linie nicht erforderlich. Ihre Breite kann daher auf 0pt gesetzt werden.
\renewcommand{\headrulewidth}{0.4pt}
%\renewcommand{\headrulewidth}{0pt}

%-----------------------------------
% Damit die hochgestellten Zahlen auch auf die Fußnote verlinkt sind (siehe Issue 169)
%-----------------------------------
\hypersetup{colorlinks=true, breaklinks=true, linkcolor=darkblack, citecolor=darkblack, menucolor=darkblack, urlcolor=darkblack, linktoc=all, bookmarksnumbered=false, pdfpagemode=UseOutlines, pdftoolbar=true}
\urlstyle{same}%gleiche Schriftart für den Link wie für den Text

%-----------------------------------
% Start the document here:
%-----------------------------------
\begin{document}

\pagenumbering{Roman}								% Seitennumerierung auf römisch umstellen
\newcolumntype{C}{>{\centering\arraybackslash}X}	% Neuer Tabellen-Spalten-Typ:
%Zentriert und umbrechbar

%-----------------------------------
% Textcommands
%-----------------------------------
%----------------------------------
%  TextCommands
%----------------------------------
%
%
%
%
%----------------------------------
%  common textCommands
%----------------------------------
% Information: OL bedeutet ohne Leerzeichen. Damit man dieses Command z. B. vor einem Komma oder vor einem anderen Zeichen verwenden kann. Dies ist ein Best-Practis von mir und hat sich sehr bewehrt.
% Allgemein hat es sich bewert alle Wörter die man häufig schreibt und wahrscheinlich falsch oder unterscheidlich schreibt, als Textcommand zu hinterlegen.
% 
%
%
\renewcommand{\symheadingname}{\langde{Symbolverzeichnis}\langen{List of Symbols}}
\newcommand{\abbreHeadingName}{\langde{Abkürzungsverzeichnis}\langen{List of Abbreviations}}
\newcommand{\headingNameInternetSources}{\langde{Internetquellen}\langen{Internet sources}}
\newcommand{\AppendixName}{\langde{Anhang}\langen{Appendix}}
\newcommand{\vglf}{\langde{Vgl.}\langen{compare}}
\newcommand{\pagef}{\langde{S. }\langen{p. }}
\newcommand{\os}{\mbox{o. S}}
\newcommand{\ojol}{\mbox{o. J.}}
\newcommand{\oj}{\ojol\ }
\newcommand{\og}{\mbox{o. g.}\ }
\newcommand{\ua}{\mbox{u. a.}\ }
\newcommand{\dah}{\mbox{d. h.}\ }
\newcommand{\zbol}{\mbox{z. B.}}
\newcommand{\zb}{\zbol\ }
\newcommand{\uamol}{unter anderem}
\newcommand{\uam}{\uamol\ }
\newcommand{\uanol}{unter anderen}%mit Leerzeichen
\newcommand{\uan}{\uanol\ }%mit Leerzeichen
\newcommand{\abbol}{Ab"-bil"-dung}
\newcommand{\abb}{\abbol\ }
\newcommand{\tabol}{Tabelle}
\newcommand{\tab}{\tabol\ }
\newcommand{\ggfol}{ggf.}
\newcommand{\ggf}{\ggfol\ }
\newcommand{\unodol}{und/oder}
\newcommand{\unod}{\unodol\ }

%----------------------------------
% project individual textCommands
%----------------------------------
\newcommand{\lehol}{Lebensmitteleinzelhandel}%Beispiel eines langen Wortes
\newcommand{\leh}{\lehol\ }


%-----------------------------------
% Titlepage
%-----------------------------------
\begin{titlepage}
	\newgeometry{left=2cm, right=2cm, top=2cm, bottom=2cm}
	\begin{center}
    \includegraphics[width=2.3cm]{abbildungen/fomLogo} \\
    \vspace{.5cm}
		\begin{Large}\textbf{\myHochschulName}\end{Large}\\
    \vspace{.5cm}
		\begin{Large}\langde{Hochschulzentrum}\langen{university location} \myHochschulStandort\end{Large}\\
		\vspace{2cm}
    \begin{Large}\textbf{\myThesisArt}\end{Large}\\
    \vspace{.5cm}
		% \langde{Berufsbegleitender Studiengang}
		% \langen{part-time degree program}\\
		% \mySemesterZahl. Semester\\
    \langde{im Studiengang}\langen{in the study course} \myStudiengang
		\vspace{1.7cm}

		%\langde{zur Erlangung des akademischen Grades}\langen{}\\
        % Oder für Hausarbeiten:
        \langde{im Rahmen der Lehrveranstaltung}\langen{}\\
    \vspace{0.5cm}
		%\begin{Large}{\myAkademischerGrad}\end{Large}\\
		% Oder für Hausarbeiten:
		%\textbf{im Rahmen der Lehrveranstaltung}\\
		\textbf{\myLehrveranstaltung}\\
		\vspace{1.8cm}
		\langde{über das Thema}
		\langen{on the subject}\\
    \vspace{0.5cm}
		\large{\textbf{\myTitel}}\\
        \normalsize- {\textbf{\mySubTitel}} -\\
		\vspace{2cm}
    \langde{von}\langen{by}\\
    \vspace{0.5cm}
    \begin{Large}{\myAutor}\end{Large}\\
	\end{center}
	\normalsize
	\vfill
    \begin{tabular}{ l l }
        \langde{Betreuer} % für Hausarbeiten
        %\langde{Erstgutachter} % für Bachelor- / Master-Thesis
        \langen{Advisor}: & \myBetreuer\\
        \langde{Matrikelnummern}
        \langen{Matriculation Number}: & \myMatrikelNr\\
        \langde{Abgabedatum}
        \langen{Submission}: & \myAbgabeDatum
    \\
    \end{tabular}
\end{titlepage}


%-----------------------------------
% Vorwort (optional; bei Verwendung beide Zeilen entkommentieren und unter Inhaltsverzeichnis setcounter entsprechend anpassen)
%-----------------------------------
%\section*{Vorwort}
Bei Bedarf erscheint vor dem Inhaltsverzeichnis ein Vorwort. Es erhält keine Kapitelnummer und wird nicht im Inhaltsverzeichnis aufgeführt (Auszug aus dem Leitfaden zur Gestaltung wissenschaftlicher Arbeiten / Dekanat ING \& IT Management, Februar 2022).

Lorem ipsum dolor sit amet, consetetur sadipscing elitr, sed diam nonumy eirmod tempor invidunt ut labore et dolore magna aliquyam erat, sed diam voluptua. At vero eos et accusam et justo duo dolores et ea rebum. Stet clita kasd gubergren, no sea takimata sanctus est Lorem ipsum dolor sit amet. Lorem ipsum dolor sit amet, consetetur sadipscing elitr, sed diam nonumy eirmod tempor invidunt ut labore et dolore magna aliquyam erat, sed diam voluptua. At vero eos et accusam et justo duo dolores et ea rebum. Stet clita kasd gubergren, no sea takimata sanctus est Lorem ipsum dolor sit amet. Lorem ipsum dolor sit amet, consetetur sadipscing elitr, sed diam nonumy eirmod tempor invidunt ut labore et dolore magna aliquyam erat, sed diam voluptua. At vero eos et accusam et justo duo dolores et ea rebum. Stet clita kasd gubergren, no sea takimata sanctus est Lorem ipsum dolor sit amet. 

Duis autem vel eum iriure dolor in hendrerit in vulputate velit esse molestie consequat, vel illum dolore eu feugiat nulla facilisis at vero eros et accumsan et iusto odio dignissim qui blandit praesent luptatum zzril delenit augue duis dolore te feugait nulla facilisi. Lorem ipsum dolor sit amet, consectetuer adipiscing elit, sed diam nonummy nibh euismod tincidunt ut laoreet dolore magna aliquam erat volutpat. 
\\[1cm]
{\myOrt}, März 2022

{\myAutor}
%\newpage

%-----------------------------------
% Inhaltsverzeichnis
%-----------------------------------
% Um das Tabellen- und Abbbildungsverzeichnis zu de/aktivieren ganz oben in Documentclass schauen
\setcounter{page}{2}
\addtocontents{toc}{\protect\enlargethispage{-20mm}}% Die Zeile sorgt dafür, dass das Inhaltsverzeichnisseite auf die zweite Seite gestreckt wird und somit schick aussieht. Das sollte eigentlich automatisch funktionieren. Wer rausfindet wie, kann das gern ändern.
\setcounter{tocdepth}{4}
\tableofcontents
\newpage

%-----------------------------------
% Abbildungsverzeichnis
%-----------------------------------
\listoffigures
\newpage
%-----------------------------------
% Tabellenverzeichnis
%-----------------------------------
\listoftables
\newpage
%-----------------------------------
% Abkürzungsverzeichnis
%-----------------------------------
% Falls das Abkürzungsverzeichnis nicht im Inhaltsverzeichnis angezeigt werden soll
% dann folgende Zeile auskommentieren.
\addcontentsline{toc}{section}{\abbreHeadingName}

\section*{\langde{Abkürzungsverzeichnis}\langen{List of Abbreviations}}

\begin{acronym}[WYSIWYG]\itemsep0pt %der Parameter in Klammern sollte die längste Abkürzung sein. Damit wird der Abstand zwischen Abkürzung und Übersetzung festgelegt
  \acro{CNNs}{Convolutional Neural Networks}
  \acro{CNN}{Convolutional Neural Network}
  \acro{FCN}{Fully Convolutional Neural Network}
  \acro{Beispiel}{Nicht verwendet, taucht nicht im Abkürzungsverzeichnis auf}
\end{acronym}
\newpage

%-----------------------------------
% Symbolverzeichnis
%-----------------------------------
% In Overleaf führt der Einsatz des Symbolverzeichnisses zu einem Fehler, der aber ignoriert werdne kann
% Falls das Symbolverzeichnis nicht im Inhaltsverzeichnis angezeigt werden soll
% dann folgende Zeile auskommentieren.
\phantomsection\addcontentsline{toc}{section}{\symheadingname}
%
%
%
%
%
%
%
% Quelle: https://www.namsu.de/Extra/pakete/Listofsymbols.pdf
% Wie ind er Quelle beschrieben führt das Verwenden von Umlauten oder ß zu einem Fehler.
% Hier werden die Symbole definiert in folgender Form:
% \newsym[Beschreibung]{Symbolbefehl}{Symbol}
\opensymdef
\newsym[Aufrechter Buchstabe]{AB}{\text{A}}
\newsym[Menge aller natuerlichen Zahlen ohne die Null]{symnz}{\mathbb{N}}
\newsym[Menge aller natuerlichen Zahlen einschliesslich Null]{symnzmn}{\mathbb{N}_{0}}
\newsym[Menge aller ganzen Zahlen]{GZ}{\mathbb{Z}}
\newsym[Menge aller rationalen Zahlen]{RatZ}{\mathbb{Q}}
\newsym[Menge aller reellen Zahlen]{RZ}{\mathbb{R}}
\closesymdef

\listofsymbols
\newpage

%-----------------------------------
% Glossar
%-----------------------------------
\printnoidxglossaries
\newpage

%-----------------------------------
% Sperrvermerk
%-----------------------------------
%\newpage
\thispagestyle{empty}

%-----------------------------------
% Sperrvermerk
%-----------------------------------
\section*{Sperrvermerk}
Die vorliegende Abschlussarbeit mit dem Titel \enquote{\myTitel} enthält unternehmensinterne Daten der Firma \myFirma . Daher ist sie nur zur Vorlage bei der FOM sowie den Begutachtern der Arbeit bestimmt. Für die Öffentlichkeit und dritte Personen darf sie nicht zugänglich sein.

\par\medskip
\par\medskip

\_\_\_\_\_\_\_\_\_\_\_\_\_\_\_\_\_\_\_\_\_\_\_\_ \hspace{1.5cm} \_\_\_\_\_\_\_\_\_\_\_\_\_\_\_\_\_\_\_\_\_\_\_\_ \\
(Ort, Datum)\hspace{4.5cm}
(Eigenhändige Unterschrift)

\newpage

%-----------------------------------
% Seitennummerierung auf arabisch und ab 1 beginnend umstellen
%-----------------------------------
\pagenumbering{arabic}
\setcounter{page}{1}

%-----------------------------------
% Kapitel / Inhalte
%-----------------------------------
% Die Kapitel werden über folgende Datei eingebunden
% Hinzugefügt aufgrund von Issue 167
%-----------------------------------
% Kapitel / Inhalte
%-----------------------------------
\section{Einleitung}
Dies soll eine \LaTeX{} -Vorlage für den persönlichen Gebrauch werden. Sie hat weder einen Anspruch auf Richtigkeit, noch auf Vollständigkeit. Die Quellen liegen auf Github zur allgemeinen Verwendung. Verbesserungen sind jederzeit willkommen. 

\subsection{Zielsetzung}
Kleiner Reminder für mich in Bezug auf die Dinge, die wir bei der Thesis beachten sollten und \LaTeX{}-Vorlage für die Thesis.

\subsection{Aufbau der Arbeit}
Kapitel 2 enthält die Inhalte des Thesis-Days und alles, was zum inhaltlichen erstellen der Thesis relevant sein könnte. Kapitel 3 wichtige Anmerkungen zu \LaTeX{}, wobei die wirklich wichtigen Dinge im Quelltext dieses Dokumentes stehen. 

\begin{figure}[H]
\begin{center}
\includegraphics[width=0.9\textwidth]{verzeichnisStruktur}
\caption{Verzeichnisstruktur der \LaTeX{}-Datein}
\end{center}
\end{figure}

\input{kapitel/kapitel_1/forschungsfrage}
\newpage
\section{Zielsetzung der Arbeit} \label{latexDetails}
\newpage
\section{Theoretischer Hintergrund
} \label{latexDetails}
Die MITRE \ac{attck} ist ein Framework zur Beschreibung von Taktiken und Techniken, die von Angreifern verwendet werden. SOAR-Systeme unterstützen Sicherheitsoperationsteams, indem sie Prozesse automatisieren und die Effizienz bei der Bedrohungserkennung und -reaktion steigern. Bisherige Forschungen zeigen die Vorteile von SOAR-Systemen in Bezug auf Zeitersparnis und Effizienz, jedoch gibt es noch Herausforderungen bei der Implementierung und Anpassung.
\newpage
\section{Methodik} \label{latexDetails}
\begin{enumerate}
\item \textbf{Laborexperimente}: In einer kontrollierten Testumgebung, bestehend aus Windows Client- und Server-Systemen sowie dem Monitoring-Tool AppSysmon, werden die häufigsten Techniken der MITRE ATT\&CK-Matrix 2023 simuliert. Hierbei werden verschiedene realistische Angriffsszenarien nachgestellt, um die Erkennbarkeit der Angriffe durch die eingesetzten Sicherheitsmechanismen zu evaluieren. Ziel dieser Experimente ist es, die Effektivität des SOAR-Systems bei der Identifizierung und Reaktion auf diese Techniken zu analysieren.
\item \textbf{Log-Analyse}: Nach der Durchführung der Laborexperimente werden die generierten Logs sorgfältig geprüft. Diese Analyse zielt darauf ab, sicherheitsrelevante Indikatoren (Indicators of Compromise, IoCs) zu identifizieren und zu bewerten. Durch die systematische Auswertung der Log-Daten wird untersucht, welche Informationen zur Erkennung von Sicherheitsvorfällen beitragen können.
\item \textbf{Automatisierungspotenzial}: Im Anschluss erfolgt eine detaillierte Untersuchung der Log-Daten, um zu ermitteln, welche Entscheidungen auf Basis der identifizierten Indikatoren automatisiert werden können. Dieser Schritt beinhaltet die Identifizierung wiederkehrender Muster und die Entwicklung von Automatisierungsregeln, die im SOAR-System implementiert werden können, um die Reaktionszeiten auf Sicherheitsvorfälle zu optimieren.
\item \textbf{Arbeitsanweisungen}: Basierend auf den Erkenntnissen aus der Log-Analyse und den identifizierten Automatisierungsmöglichkeiten werden umfassende Workflows und Handlungsanweisungen erstellt. Diese Dokumentationen dienen als Leitfaden für Sicherheitsteams und sollen die Implementierung der automatisierten Prozesse unterstützen. Die Workflows werden so gestaltet, dass sie sowohl die manuelle als auch die automatisierte Reaktion auf unterschiedliche Szenarien abdecken.
\item \textbf{Bewertung}:Abschließend erfolgt ein Vergleich der Ergebnisse, bei dem die Effizienz und Effektivität der manuellen Prozesse gegen die automatisierten Prozesse abgewogen werden. Diese Bewertung hat zum Ziel, die Vor- und Nachteile des SOAR-Systems zu ermitteln, insbesondere hinsichtlich der Geschwindigkeit, Genauigkeit und Zuverlässigkeit der Reaktionen auf Sicherheitsvorfälle. Die gewonnenen Erkenntnisse sollen dazu beitragen, Verbesserungspotenziale aufzuzeigen und die Implementierung von SOAR-Systemen in der Praxis zu unterstützen.
\end{enumerate}
%\newpage
\section{Zielsetzung der Arbeit} \label{latexDetails}
%\newpage
\section{Schlussbetrachtung}
(Umfang ca. 0,5 - 1 Seite)

\subsection{Zusammenfassung}

\subsection{Fazit}

\subsection{Kritische Reflexion}

\subsection{Ausblick}



%-----------------------------------
% Apendix / Anhang
%-----------------------------------
\newpage
\section*{\AppendixName} %Überschrift "Anhang", ohne Nummerierung
\addcontentsline{toc}{section}{\AppendixName} %Den Anhang ohne Nummer zum Inhaltsverzeichnis hinzufügen

\begin{appendices}
% Nachfolgende Änderungen erfolgten aufgrund von Issue 163
\makeatletter
\renewcommand\@seccntformat[1]{\csname the#1\endcsname:\quad}
\makeatother
\addtocontents{toc}{\protect\setcounter{tocdepth}{0}} %
	\renewcommand{\thesection}{\AppendixName\ \arabic{section}}
	\renewcommand\thesubsection{\AppendixName\ \arabic{section}.\arabic{subsection}}
	\section{Beispielanhang}\label{Beispielanhang}
Dieser Abschnitt dient nur dazu zu demonstrieren, wie ein Anhang aufgebaut seien kann.
\subsection{Weitere Gliederungsebene}
Auch eine zweite Gliederungsebene ist möglich.
\section{Bilder}
Auch mit Bildern.
Diese tauchen nicht im Abbildungsverzeichnis auf.
\begin{figure}[H]
    \centering
    \caption[]{Beispielbild}
	\label{fig:Beispielbild}
    \includegraphics[width=1\textwidth]{verzeichnisStruktur}
\end{figure}
\end{appendices}
\addtocontents{toc}{\protect\setcounter{tocdepth}{2}}

%-----------------------------------
% Literaturverzeichnis
%-----------------------------------
\newpage

% Die folgende Zeile trägt ALLE Werke aus literatur.bib in das
% Literaturverzeichnis ein, egal ob sie zietiert wurden oder nicht.
% Der Befehl ist also nur zum Test der Skripte sinnvoll und muss bei echten
% Arbeiten entfernt werden.
%\nocite{*}

%\addcontentsline{toc}{section}{Literatur}

% Die folgenden beiden Befehle würden ab dem Literaturverzeichnis wieder eine
% römische Seitennummerierung nutzen.
% Das ist nach dem Leitfaden nicht zu tun. Dort steht nur dass 'sämtliche
% Verzeichnisse VOR dem Textteil' römisch zu nummerieren sind. (vgl. S. 3)
%\pagenumbering{Roman} %Zähler wieder römisch ausgeben
%\setcounter{page}{4}  %Zähler manuell hochsetzen

% Ausgabe des Literaturverzeichnisses

% Keine Trennung der Werke im Literaturverzeichnis nach ihrer Art
% (Online/nicht-Online)
%\begin{RaggedRight}
%\printbibliography
%\end{RaggedRight}

% Alternative Darstellung, die laut Leitfaden genutzt werden sollte.
% Dazu die Zeilen auskommentieren und folgenden code verwenden:

% Literaturverzeichnis getrennt nach Nicht-Online-Werken und Online-Werken
% (Internetquellen).
% Die Option nottype=online nimmt alles, was kein Online-Werk ist.
% Die Option heading=bibintoc sorgt dafür, dass das Literaturverzeichnis im
% Inhaltsverzeichnis steht.
% Es ist übrigens auch möglich mehrere type- bzw. nottype-Optionen anzugeben, um
% noch weitere Arten von Zusammenfassungen eines Literaturverzeichnisse zu
% erzeugen.
% Beispiel: [type=book,type=article]
\printbibliography[nottype=online,heading=bibintoc,title={\langde{Literaturverzeichnis}\langen{Bibliography}}]

% neue Seite für Internetquellen-Verzeichnis
\newpage

% Laut Leitfaden 2018, S. 14, Fussnote 44 stehen die Internetquellen NICHT im
% Inhaltsverzeichnis, sondern gehören zum Literaturverzeichnis.
% Die Option heading=bibintoc würde die Internetquelle als eigenen Eintrag im
% Inhaltsverzeicnis anzeigen.
%\printbibliography[type=online,heading=bibintoc,title={\headingNameInternetSources}]
\printbibliography[type=online,heading=subbibliography,title={\headingNameInternetSources}]

\newpage
\pagenumbering{gobble} % Keine Seitenzahlen mehr

%-----------------------------------
% Ehrenwörtliche Erklärung
%-----------------------------------
\section*{%
	\langde{Ehrenwörtliche Erklärung}
	\langen{Declaration in lieu of oath}}
\langde{Hiermit versichere ich, dass die vorliegende Arbeit von mir selbstständig und ohne unerlaubte Hilfe angefertigt worden ist, insbesondere dass ich alle Stellen, die wörtlich oder annähernd wörtlich aus Veröffentlichungen entnommen sind, durch Zitate als solche gekennzeichnet habe. Ich versichere auch, dass die von mir eingereichte schriftliche Version mit der digitalen Version übereinstimmt. Weiterhin erkläre ich, dass die Arbeit in gleicher oder ähnlicher Form noch keiner Prüfungsbehörde/Prüfungsstelle vorgelegen hat. Ich erkläre mich damit \textcolor{red}{einverstanden/nicht einverstanden}, dass die Arbeit der Öffentlichkeit zugänglich gemacht wird. Ich erkläre mich damit einverstanden, dass die Digitalversion dieser Arbeit zwecks Plagiatsprüfung auf die Server externer Anbieter hochgeladen werden darf. Die Plagiatsprüfung stellt keine Zurverfügungstellung für die Öffentlichkeit dar.}
\langen{I hereby declare that I produced the submitted paper with no assistance from any other party and without the use of any unauthorized aids and, in particular, that I have marked as quotations all passages which are reproduced verbatim or near-verbatim from publications. Also, I declare that the submitted print version of this thesis is identical with its digital version. Further, I declare that this thesis has never been submitted before to any examination board in either its present form or in any other similar version. I herewith \textcolor{red}{agree/disagree} that this thesis may be published. I herewith consent that this thesis may be uploaded to the server of external contractors for the purpose of submitting it to the contractors’ plagiarism detection systems. Uploading this thesis for the purpose of submitting it to plagiarism detection systems is not a form of publication.}


\par\medskip
\par\medskip

\vspace{5cm}

\begin{table}[H]
	\centering
	\begin{tabular*}{\textwidth}{c @{\extracolsep{\fill}} ccccc}
		\myOrt, \the\day.\the\month.\the\year
		&
		% Hinterlege deine eingescannte Unterschrift im Verzeichnis /abbildungen und nenne sie unterschrift.png
		% Bilder mit transparentem Hintergrund können teils zu Problemen führen
		\includegraphics[width=0.35\textwidth]{unterschrift}\vspace*{-0.35cm}
		\\
		\rule[0.5ex]{12em}{0.55pt} & \rule[0.5ex]{12em}{0.55pt} \\
		\langde{(Ort, Datum)}\langen{(Location, Date)} & \langde{(Eigenhändige Unterschrift)}\langen{(handwritten signature)}
		\\
	\end{tabular*} \\
\end{table}

\end{document}
