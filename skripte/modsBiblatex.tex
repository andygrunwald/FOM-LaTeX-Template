% Optionen für Biblatex
\ExecuteBibliographyOptions{%
  giveninits=false,
  isbn=true,
  url=true,
  doi=false,
  eprint=false,
  maxbibnames=7,      % Alle Autoren (kein et al.)
  maxcitenames=2,     % et al. ab dem 3. Autor
  backref=false,      % Rückverweise auf Zitatseiten
  bibencoding=utf8,   % wenn .bib in utf8, sonst ascii
  bibwarn=true        % Warnung bei fehlerhafter bib-Datei
}%

% et al. an Stelle von u.a.
\DefineBibliographyStrings{ngerman}{
   andothers = {{et\,al\adddot}},                   % TODO et.al. prüfen
}

% Klammern um das Jahr in der Fußnote
\renewbibmacro*{cite:labelyear+extrayear}{%
  \iffieldundef{labelyear}                          % TODO footcite anpassen
    {}
    {\printtext[bibhyperref]{%
       \mkbibparens{%
         \printfield{labelyear}%
         \printfield{extrayear}}}}}

\DeclareNameFormat{last-first}{%
  \iffirstinits
    {\usebibmacro{name:family-given}
        {\namepartfamily}
        {\namepartgiveni}
        {\namepartprefix}
        {\namepartsuffix}
    }
    {\usebibmacro{name:family-given}
        {\namepartfamily}
        {\namepartgiven}
        {\namepartprefix}
        {\namepartsuffix}
    }%
  \usebibmacro{name:andothers}}


% Alternative Notation der Fußnoten
% Zeigt sowohl den Nachnamen als auch den Vornamen an
% Beispiel: \fullfootcite[Vgl. ][Seite 5]{Tanenbaum.2003}
\DeclareCiteCommand{\fullfootcite}[\mkbibfootnote]    % TODO Autor nur Inital als Vornamen!
  {\printtext{Vgl.\isdot}}                            % TODO Bsp.-PDF anpassen
  {\usebibmacro{citeindex}%                           % TODO Zahlen [] raus
    \addspace\textit{\printnames[sortname][1-1]{author}}%
    \addcomma\addspace\printfield{shorttitle}\addcomma\space
    \addspace\printfield{year}\addcomma\space
  }
  {\printtext{S.\isdot}\space}
  {\printfield{pages}}

%Autoren (Nachname, Vorname)
\DeclareNameAlias{default}{family-given}
%\DeclareNameAlias{sortname}{family-given}

%Reihenfolge von publisher, year, address verändern
% Achtung, bisher nur für den Typ @book definiert

%% Definiert @Book Eintrag         % TODO pages f. ff.
\DeclareBibliographyDriver{book}{%
  \textit{\printnames{author}}%
  \newunit\space
  \printtext{$($}%
  \printfield{shorttitle}%         % TODO durch Stichwort ersetzen
  \setunit*{\addcomma\space}%
  \printfield{year}%
  \printtext{$)$}%
  \setunit*{\addcolon\space}%
  \printfield{title}%
  \setunit*{\addcomma\space}%
  \printfield{edition}%
  \setunit*{\addcomma\space}%
  \printlist{location}%           % TODO falls mehr als drei Orte u.a.
  \setunit*{\addcolon\space}%
  \printlist{publisher}%
  \setunit*{\space}%
  \printfield{year}%
  %\newunit\newblockpunct
}

%% Definiert @Online Eintrag    % TODO Overhaule needed
\DeclareBibliographyDriver{online}{%
  \printnames{author}%
  \newunit\newblockpunct
  \printfield{title}%
  \setunit*{,\space}%
  %\newunit\newblock
  \printfield{url}%
  \setunit*{,\space Erscheinungsjahr:\space}%
  \printfield{year}%
  \setunit*{,\space Aufruf am:\space}%
  \printfield{note}%
}
  
%% Definiert @Article Eintrag
\DeclareBibliographyDriver{article}{%
  \printnames{author}%
  \newunit\newblockpunct
  \printfield{title}%
  \setunit*{.\space In:\space}%
  %\newunit\newblock
  \usebibmacro{journal}%
  \setunit*{\space (}%
  \printfield{year}\newunit{)}%
}

%Doppelpunkt nach dem letzten Autor
%\renewcommand*{\labelnamepunct}{\addcolon\addspace}

%Komma an Stelle des Punktes, nach Autor
%\renewcommand*{\newunitpunct}{\addcomma\space}

%Leerzeichen an Stelle des Punktes, nach Autor
\renewcommand*{\newunitpunct}{\space}

%Autoren durch Semikolon trennen
%\newcommand*{\bibmultinamedelim}{\addsemicolon\space}%
%\newcommand*{\bibfinalnamedelim}{\addsemicolon\space}%
%\AtBeginBibliography{%
%  \let\multinamedelim\bibmultinamedelim
%  \let\finalnamedelim\bibfinalnamedelim
%}

%Titel nicht kursiv anzeigen 
\DeclareFieldFormat{title}{#1\isdot}
%\finentry % dot