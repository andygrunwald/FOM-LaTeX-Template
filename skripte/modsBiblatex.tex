% Opptionen für Biblatex
% Note Cubasepp laut leitfaden:
% - ist es besser den vornamen in kurzform anzugeben
% - bücher ohne isbn anzugeben 
\ExecuteBibliographyOptions{%
giveninits=true,
isbn=false, 
url=true, 
doi=false, 
eprint=false,
maxbibnames=7, % Alle Autoren (kein et al.)
maxcitenames=2, % et al. ab dem 3. Autor
backref=false, % Rückverweise auf Zitatseiten
bibencoding=utf8, % wenn .bib in utf8, sonst ascii
bibwarn=true, % Warnung bei fehlerhafter bib-Datei
}%

% et al. an Stelle von u.a.
\DefineBibliographyStrings{ngerman}{ 
   andothers = {{et\,al\adddot}},             
}

% Klammern um das Jahr in der Fußnote
\renewbibmacro*{cite:labelyear+extrayear}{% 
  \iffieldundef{labelyear} 
    {} 
    {\printtext[bibhyperref]{% 
       \mkbibparens{% 
         \printfield{labelyear}% 
         \printfield{extrayear}}}}}

\DeclareNameFormat{last-first}{%
  \iffirstinits
    {\usebibmacro{name:family-given}
        {\namepartfamily}
        {\namepartgiveni}
        {\namepartprefix}
        {\namepartsuffix}
    }
    {\usebibmacro{name:family-given}
        {\namepartfamily}
        {\namepartgiven}
        {\namepartprefix}
        {\namepartsuffix}
    }%
  \usebibmacro{name:andothers}}

% Alternative Notation der Fußnoten 
% Zeigt sowohl den Nachnamen als auch den Vornamen an
% Beispiel: \fullfootcite[Vgl. ][Seite 5]{Tanenbaum.2003} 
\DeclareCiteCommand{\fullfootcite}[\mkbibfootnote]
  {\usebibmacro{prenote}}
  {\usebibmacro{citeindex}%
    \printnames[sortname][1-1]{author}%
    \addspace (\printfield{year})}
  {\addsemicolon\space}
  {\usebibmacro{postnote}}

%Autoren (Nachname, Vorname)
\DeclareNameAlias{default}{family-given}

%% Note Cubasepp: Laut Leitfaden sind: 
% - die Titel (egal welche) ohne Anführungszeichen
% - Anpassung von @Inbook elementen
% - Anpassung von Online Quellen
\clareFieldFormat[article,book,inbook,inproceedings]{title}{{#1}}
\DeclareFieldFormat[inproceedings]{booktitle}{{#1}}
\DeclareFieldFormat[inbook]{chapter}{{#1}}

% Laut Leitfaden...
\DeclareFieldFormat{urldate}{\addcomma\space\bibstring{urlseen}\space#1}
\DefineBibliographyStrings{german}{%
	urlseen = {Abruf am},
}

%Reihenfolge von publisher, year, address verändern
%Note Cubasepp - TodO: Achtung die ab jetzt verwendeten Änderungen sind dem Leitfaden zum Opfer gefallen.

% Definiert @Book Eintrag
% Note Cubasepp: laut leitfaden ...
\DeclareBibliographyDriver{book}{%
  \printnames{author}%
  \setunit*{\space (}%
  \printfield{year}\newunit{)}%
  \newunit\addcolon\space
  \printfield{title}%
  \setunit*{,\space}% 
  \printlist{location}%
  \setunit*{\space}% 
  \printfield{year}%
  \finentry}

% Add @Inbook style
% Note Cubasepp: laut leitfaden ...
\DeclareBibliographyDriver{inbook}{%
  \printnames{author}%
  \setunit*{\space (}%
  \printfield{year}\newunit{)}%
  \newunit\addcolon\space
  \printfield{chapter}%
  \setunit*{, in:\space}%
  \printnames{editor}\newunit{\space (Hrsg.)}%
  \setunit*{, \space}%
  \printfield{title}%
  \setunit*{,\space}% 
  \printlist{location}%
  \setunit*{\space}% 
  \printfield{year}%
  \setunit*{, \space}%
  \printfield{pages}%
  \finentry}

% Definiert @Online Eintrag
% Note Cubasepp: laut leitfaden ...
\DeclareBibliographyDriver{online}{%
  \printnames{author}%
  \newunit\newblockpunct
  \printfield{title}%
  \setunit*{,\space}%
  %\newunit\newblock
  \printfield{url}%
  \setunit*{,\space Erscheinungsjahr:\space}%
  \printfield{year}%
  \setunit*{,\space Aufruf am:\space}%
  \printfield{note}%
  \finentry}

% Definiert @Article Eintrag
% Note Cubasepp: laut leitfaden ...
\DeclareBibliographyDriver{article}{%
  \printnames{author}%
  \setunit*{\space (}%
  \printfield{year}\newunit{)}%
  \newunit\addcolon\space
  \printfield{title}%
  \setunit*{.\space In:\space}%
  %\newunit\newblock
  \usebibmacro{journal}%
  \setunit*{\space }%
  \printfield{year}%
  \setunit*{, \space}%
  \setunit*{, Nr. \space}%
  \printfield{number}\newunit{, \space}%
  \printfield{pages}%
  \finentry}  

% Definiert @Inproceeedings Eintrag
% (booktitle != title) - das ist Grund zum unterschied @Article
% Note Cubasepp: laut leitfaden sieht es so aus
\DeclareBibliographyDriver{inproceedings}{%
  \printnames{author}%
  \setunit*{\space (}%
  \printfield{year}\newunit{)}%
  \newunit\addcolon\space
  \printfield{title}%
  \setunit*{.\space In:\space}%
  %\newunit\newblock
  \printfield{booktitle}%
  \setunit*{\space }%
  \printfield{year}%
  \setunit*{, \space}%
  \setunit*{, Nr. \space}%
  \printfield{number}\newunit{, \space}%
  \printfield{pages}%
  \finentry}  


%Doppelpunkt nach dem letzten Autor
\renewcommand*{\labelnamepunct}{\addcolon\addspace }

%Komma an Stelle des Punktes
\renewcommand*{\newunitpunct}{\addcomma\space}

%Autoren durch Semikolon trennen
\newcommand*{\bibmultinamedelim}{\addsemicolon\space}% 
\newcommand*{\bibfinalnamedelim}{\addsemicolon\space}% 
\AtBeginBibliography{% 
  \let\multinamedelim\bibmultinamedelim 
  \let\finalnamedelim\bibfinalnamedelim 
}

%Titel nicht kursiv anzeigen 
\DeclareFieldFormat{title}{#1\isdot}
