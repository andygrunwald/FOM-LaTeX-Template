\newpage
\section{Theoretische Grundlagen} \label{theorie}
In diesem Kapitel werden anhand einer Literaturanalyse die theoretischen Grundlagen für die nachfolgenden praktischen Ausarbeitung geschaffen...

%INS Vorwort?: Es wird immer Gesichtserkennung mithilfe von KI gemeint, wenn von Gesichtserkennung genannt wird. Gesichtserkennung durch Menschen wird expliziert  

\subsection{Gesichtserkennung durch KI}

\subsubsection{Künstliche Intelligenz}
% Was ist eine KI -> Erklärung und Historie
\acf*{KI} bezeichnet die Nachbildung der intelligenten Verhaltensweisen von Menschen durch Maschinen\footcite[Vgl. ][]{copeland_artificial_2022}. Es gibt mehrere Definitionen intelligentes Verhalten von Menschen zu erklären \footcite[Vgl. ][]{paas_kunstliche_2020}. Gradner \footcite[Vgl. ][]{gardner_frames_1984} entwickelte eine Theorie, die die menschliche Intelligenz in acht Bereiche einteilt. Die Bereiche der Intelligenz beschreiben \zb das logisch-mathematische denken, das zwischenmenschliche oder emotionale interagieren und die bildlich-räumliche Vorstellungskraft. \ac{KI} wird mittlerweile in den meisten Bereiche eingesetzt, wobei die bildlich-räumliche Intelligenz für diese Arbeit relevant ist. Um die Intelligenz eines Computers zu ermitteln, wurde der Turing-Test von Alan Turing bereits 1950 erfunden\footcite[Vgl. ][]{moor_turing_2003}. Der Test ist so aufgebaut, dass eine prüfende Person anonymisiert mit einem menschlichen Fachkundigen und einem Computer kommuniziert. Ein Computer gilt als intelligent, wenn die prüfende Person nach längerem Kontakt mit dem Computer und dem Menschen, die Seiten nicht zuordnet. % Beispiel nennen?
% was einfach und starke Künstliche Intelligenz ?


\subsubsection{Die künstliche Gesichtserkennung} 
Der Grundbestandteil der künstlichen Gesichtserkennung ist das Erkennen durch Algorithmen von Gesichter auf Bilder. Es werden verschiedenen automatisierte Verfahren der Bilderkennung eingesetzt, mit denen Objekte in Bildern identifiziert werden können\footcite[Vgl. ][Seite 119]{paas_kunstliche_2020}. Neben dem Erkennen eines Gesichtes beschäftigt sich die Gesichtserkennung mit der zuordung von Personen zu Gesichter. Ein Vergleich zwischen zwei oder mehreren Gesichtern. Ein zweiter bedeutsamer Bestandteil nach dem Erkennen eines Gesichts ist die Analyse. In der Analyse eines Gesichts geht es um das Erkennen von Eigenschaften. Es gibt weitere Einsatz- und Kombinationsmöglichkeiten die in Rahmen dieser Arbeit nicht beleuchtet werden. 


\subsubsection{Deep-Learning}
Seit 1990 bewährten sich Gesichtserkennungsalgorithmen, die Gesichter in Bilder zuverlässig identifizieren, die unter kontrollierten Bedingungen aufgenommen wurden\footcite[Vgl. ][]{ranjan_deep_2018}\footcite[Vgl. ][]{noauthor_face_nodate}. Oft sind diese kontrollierten Bedingungen nicht gegeben. Bei leicht kompromittierten Bildern sinkt die Erkennungsrate stark. Zum einen abhängige von dem individuellen Gesicht wie Frisur, Make-UP, Brille und Maske. Zum anderen durch äußerliche Einflüsse, wie die Bildqualität, stark schwankende Lichtverhältnisse oder die Position des Gesichts auf dem Bild\footcite[Vgl. ][Seite 150]{paas_kunstliche_2020} Die Erkennungsrate der Gesichtserkennung konnte durch Deep Learning verbessert werden, was vermehrt ab 2010 zum Einsatz kam\footcite[Vgl. ][]{chen_unconstrained_2018}\footcite[Vgl. ][]{ranjan_all--one_2021}.
Deep Learning ist ein Synonym für \ac{DCNN} ein Teilgebiet des Machine Learnings. Es werden \ac{CNN} (deutsch künstlich neuronale Netze) eingesetzt, die von den neuronalen Netzen des menschlichen Gehirns inspiriert sind\footcite[Vgl. ][S. 2354]{rawat_deep_2017}. Dieses künstlich neuronale Netzwerk besteht im Allgemeinen aus Modulen von Convolutional Layern und Pooling Layer, die zu mehreren Layern gruppiert und verbunden werden. Ein \ac{CNN} wird als \ac{DCNN} bezeichnet, wenn sich die beschriebene Struktur beliebig oft wiederholt und das Netzwerk „tief“ wird. %kurz Funktionsweise erläutern %ein Beispiel?


\subsubsection{Erkennen und Ausrichtung von Gesichter} 
Die aktuellen Algorithmen zur Erkennung von Gesichter verwenden ein Detektor System\footcite[Vgl. ][S. 66]{ranjan_deep_2018}\footcite[Vgl. ][]{noauthor_face_nodate}. Zuerst erfolgt die Erkennung von Gesichtern auf Bildern. Der Detektor erkennt Gesichter mit unterschiedlichen Posen, Verdeckungen oder Frisuren auf einem Bild mit unterschiedlicher Qualität oder Größe zuverlässig. Darauf werden markante Punkte wie die Augenmitte, die Nasenspitze oder den Mundwinkel in einem Gesicht lokalisiert. Nach der Erkennung liegt das aus dem Bild extrahierten Gesicht in einer normalisierten Form ohne Hintergrund zur Weiterverarbeitung bereit. 

\subsubsection{Identifikation von Personen anhand Gesichter}





\subsubsection{Anaylse von Gesichtsmerkmalen}
Die Analyse von Gesichtsmerkmalen 





\subsection{Unterrepräsentierte Personengruppen in der Gesichtserkennung}
% Warum sind die Gruppen unterrepräsentiert?

% Buch Künstliche Intelligenz S152


\subsection{Einfluss durch Corona in der KI}
% Verdeckung generell und Auswirkungen durch Corona auf die Forschung, hat die Entwicklung sich verbessert? 