\newpage
\section{Theoretische Grundlagen}
Siehe auch Wissenschaftliches Arbeiten~\footcite[Vgl. ][Seite 1]{Balzert.2008}. Damit sollten alle wichtigen Informationen abgedeckt sein ;-)

\subsection{Codequalität}
Codequalität ist ein Teilaspekt der Softwarequalität und somit essentiell für die Softwareentwicklung. Folgende Parameter beschreiben, was die Qualität eines Programmcodes ausmacht:

1) Lesbarkeit
\begin{itemize}
	\item Der Programmcode ist für einen Entwickler einfach zu lesen und kann diesen ohne großes einarbeiten in der der Dokumentation verstehen
	
	\item Die Benennung der Klassen-, Methoden- und Variablen haben ein einheitliches Schema und nutzen Idealerweise einen Standard wie zum Beispiel den PSR-2 Standard.
	
	\item 
		
\end{itemize}

2) Testbarkeit
\begin{itemize}	
	\item Der geschriebene Quellcode sollte so aufgebaut sein, dass dieser automatisiert in kleinen Modulen/Units getestet werden kann sofern nicht schon Test- beziehungsweise Verhaltensgetrieben entwickelt wurde.
	
\end{itemize}

3) Variabilität
\begin{itemize}	
	\item Der Quellcode muss einfach erweiter- und anpassbar sein um die Möglichkeit für neue Funktionen in der Software nicht zu verwehren.
	
\end{itemize}

4) Flexibilität
\begin{itemize}	
	\item Abhängigkeiten im eigenen Programmcode-Fundament und den Implementierungen sollten gering gehalten werden.
	
	\item Es sollten keine festen Konstanten im Quellcode vorhanden sein, z.B. 19 \% Mehrwertsteuer.
	
	\item Statisch programmierte Voraussetzungen über Datenstrukturen, Klassen oder Datengröße machen den Quellcode schwieriger adaptierbar.
	
\end{itemize}

5) Performance
\begin{itemize}	
	\item Die Verwendung der vorhandenen Systemressourcen durch das entwickelte Programm sollte unnötige Verwendungen der \nomenclature{CPU}{Central Processing Unit}CPU oder dem Hauptspeicher vermeiden.
	
\end{itemize}

Da diese Parameter eine große Auswirkung auf die Qualität der entwickelten Software hat, müssen diese während des gesamten Prozesses kontrolliert werden. Idealerweise geschieht dies durch geeignete Tools wie z.B. !"!"!"!"!"!"!"!

Um die bestmögliche Codequalität zu garantieren, existieren verschiedene Ansätze den Softwareentwickler zu unterstützen. Schon bei der Analyse der Anforderungen und dem
Design für eine Funktion, versuchen verhaltens- und testgetriebene Ansätze den Entwickler entscheidend zu unterstützen.

\subsection{Testgetriebene Entwicklung}
Wenn eine neue Funktionalität in einem Programm implementiert bzw. eine Funktionalität angepasst und erweitert werden soll, wie stellt man sicher, dass es im Nachhinein
zu keinerlei Problemen kommt? Die Funktionalität per Hand zu testen ist aus mehreren Gründen nicht vorteilhaft.

Die testgetriebene Entwicklung versucht dieses Problem zu beheben. Um dieses Ziel zu erreichen, wird zu Erst mit einem Test die Funktionalität spezifiziert. Nach der
Fertigstellung des Tests wird der Programmcode entwickelt. Das führt dazu, dass nun die komplette Funktionalität überprüft werden kann. Wenn unerwünschte Seiteneffekte 
entstehen, die auf Grund einer Codeänderung an einer anderen Stelle auftreten, können sie nun durch die Tests herausgefunden und dadurch
behoben werden.

\subsection{Verhaltensgetriebene Entwicklung}
BDD wurde ursprünglich 2003 von Dan North als Weiterentwicklung von TDD bekannt gemacht. 6

Dan North führte dabei syntaktische Konventionen für Unit-Tests ein. Als „Unit-Tests“ bezeichnet man Überprüfungen ob Komponenten wie gewünscht funktionieren. Er
entwickelte „JBehave“ als Ersatz für „JUnit“, das alle verwandten Wörter von „Test“ mit dem Wort „Verhalten“ ersetzt hat. „JUnit“ ist ein Framework zum Testen von Java-
Programmen welches von „JBehave“ durch veränderte Namenskonventionen abgelöst wurde.

Wieso führte Dan North eine Vokabular Umstellung durch? Edward Sapir und Benjamin Whorf bildeten eine Hypothese die aussagt, dass die Sprache, die wir nutzen, unser 
Denken beeinflusst 7 . Wollen wir unsere Denkweise verändern, hilft es demzufolge nach, die Sprache zu verändern.

Die Testgetriebene Entwicklung führte dazu, dass viele Entwickler den Entwicklungszyklus nicht optimal verwendet haben. Deswegen kam Dan North auf die Idee, durch
Namenskonventionen das Verhalten in den Mittelpunkt zu rücken. Die Basis von BDD sind flexible Methoden, die darauf abzielen, Teams mit wenig Erfahrung in agiler 
Softwareentwicklung, den Einstieg zugänglicher und effizienter zu gestalten.


\subsection{Arten von automatisierten Tests}\label{arten-von-tests} 
\subsubsection{Unit-Tests}

\subsubsection{Integrationtests}
\subsubsection{GUI-Tests}
\subsubsection{Integrationstests}
Integrationstest sind Black Box System Tests. Jeder Integrationstest repräsentiert ein zu erwartendes
Ergebnis von dem System. Integrationstest werden vom Testmanager erstellt, bezogen auf neue
Features in diesem Release. Eine Feature kann mehrere Integrationstests haben. Ein Feature ist nicht
als vollständig angesehen bis er seine Integrationstests bestanden hat. 

\subsubsection{Regressionstests}
In der Entwicklung sollte nach jeder Änderung ein Regressionstest ausgeführt werden. Bei Regressionstest
müssen Sie darauf achten, dass 1) Sie immer die gleichen Tests ausführen, wenn Sie einen
bestimmen Code-Abschnitt testen und 2) das betreffende Tests die Akzeptanzkriterien der jeweiligen
Anforderung abgleicht. Wenn nun später aber Jenkins in den Testprozess integriert würde, könnte
man sich diese Arbeit sparen. Mit Jenkins würde es dann möglich sein, die Regressionstest regelmäßig
auszuführen, etwas nach jeder Codeänderung oder einmal pro Nacht (Nightly Build). Falls Probleme auftrenten sollte, würde das Team per Email, HipChat oder ähnlichen Kommunikationswege informiert.



\subsection{Automatisches Deployment von Software}
