\newpage
\section{Theoretische Grundlagen} \label{theorie}
In diesem Kapitel werden anhand einer Literaturanalyse die theoretischen Grundlagen für die nachfolgenden praktischen Ausarbeitung geschaffen...

INS Vorwort?: Es wird immer Gesichtserkennung mithilfe von KI gemeint, wenn von Gesichtserkennung genannt wird. Gesichtserkennung durch Menschen wird expliziert  

\subsection{Gesichtserkennung durch KI}

\subsubsection{Künstliche Intelligenz}
% Was ist eine KI -> Erklärung und Historie
\acf*{KI} bezeichnet die Nachbildung der intelligenten Verhaltensweisen von Menschen durch Maschinen\footcite[Vgl. ][]{copeland_artificial_2022}. Es gibt mehrere Definitionen intelligentes Verhalten von Menschen zu erklären \footcite[Vgl. ][]{paas_was_2020}. Gradner \footcite[Vgl. ][]{gardner_frames_1984} entwickelte eine Theorie, welche die menschliche Intelligenz in 8 Bereiche einteilt. Die Bereiche der Intelligenz beschreiben \zb das logisch-mathematische denken, das zwischenmenschliche oder emotionale interagieren und die bildlich-räumliche Vorstellungskraft. \ac{KI} wird mittlerweile in vielen dieser Bereiche eingesetzt, wobei die bildlich-räumliche Intelligenz für diese Arbeit relevant ist. Um die Intelligenz eines Computer zu ermitteln, wurde der Turing-Test von Alan Turing bereits 1950 erfunden\footcite[Vgl. ][]{moor_turing_2003}. Der Test ist so aufgebaut, dass ein Prüfer anymonisert mit einem menschlichen Experten und einem Computer kommuniziert. Ein Computer gilt als intelligent, wenn der Prüfer nach längerem Kontakt mit dem Computer und dem Menschen, die Seiten nicht zuordnet. 

\subsubsection{Gesichtskennung}
In der Gesichtserkennung werden die verschiedenen automatisierte Verfahren der Bilderkennung eingesetzt, mit denen Objekte in Bildern identifiziert werden können. Dazu gehört einerseits das Klassifizieren von Bildobjekten, andererseits das Bestimmen ihrer Position im Bild\footcite[Vgl. ][Seite 119]{paas_was_2020}. Damit eine Gesichtserkennung Gesichter erkennt, muss dieser erst Trainingsdaten erhalten und "trainiert" werden.


\subsubsection{Verwendeter Algorithmus}



\subsection{Unterrepräsentierte Personengruppen in der Gesichtserkennung}
Das ermitteln der Gesichter ist dabei mit verschiedenen Schwierigkeiten verbunden. Zum einem von der Person abhängige Schwierigkeiten wie Frisur, Make-UP, Brille und Maske. Zum anderen durch äußerliche Einflüsse, wie die Bildqualität, start schwankende Lichtverhältnisse oder die Position des Gesichts auf dem Bild\footcite[Vgl. ][Seite 150]{paas_was_2020}


\subsection{Einfluss durch Corona in der KI}
