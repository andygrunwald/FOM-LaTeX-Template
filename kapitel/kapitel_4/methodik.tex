\newpage
\section{Methodik} \label{latexDetails}
\begin{enumerate}
\item \textbf{Laborexperimente}: In einer kontrollierten Testumgebung, bestehend aus Windows Client- und Server-Systemen sowie dem Monitoring-Tool AppSysmon, werden die häufigsten Techniken der MITRE ATT\&CK-Matrix 2023 simuliert. Hierbei werden verschiedene realistische Angriffsszenarien nachgestellt, um die Erkennbarkeit der Angriffe durch die eingesetzten Sicherheitsmechanismen zu evaluieren. Ziel dieser Experimente ist es, die Effektivität des SOAR-Systems bei der Identifizierung und Reaktion auf diese Techniken zu analysieren.
\item \textbf{Log-Analyse}: Nach der Durchführung der Laborexperimente werden die generierten Logs sorgfältig geprüft. Diese Analyse zielt darauf ab, sicherheitsrelevante Indikatoren (Indicators of Compromise, IoCs) zu identifizieren und zu bewerten. Durch die systematische Auswertung der Log-Daten wird untersucht, welche Informationen zur Erkennung von Sicherheitsvorfällen beitragen können.
\item \textbf{Automatisierungspotenzial}: Im Anschluss erfolgt eine detaillierte Untersuchung der Log-Daten, um zu ermitteln, welche Entscheidungen auf Basis der identifizierten Indikatoren automatisiert werden können. Dieser Schritt beinhaltet die Identifizierung wiederkehrender Muster und die Entwicklung von Automatisierungsregeln, die im SOAR-System implementiert werden können, um die Reaktionszeiten auf Sicherheitsvorfälle zu optimieren.
\item \textbf{Arbeitsanweisungen}: Basierend auf den Erkenntnissen aus der Log-Analyse und den identifizierten Automatisierungsmöglichkeiten werden umfassende Workflows und Handlungsanweisungen erstellt. Diese Dokumentationen dienen als Leitfaden für Sicherheitsteams und sollen die Implementierung der automatisierten Prozesse unterstützen. Die Workflows werden so gestaltet, dass sie sowohl die manuelle als auch die automatisierte Reaktion auf unterschiedliche Szenarien abdecken.
\item \textbf{Bewertung}:Abschließend erfolgt ein Vergleich der Ergebnisse, bei dem die Effizienz und Effektivität der manuellen Prozesse gegen die automatisierten Prozesse abgewogen werden. Diese Bewertung hat zum Ziel, die Vor- und Nachteile des SOAR-Systems zu ermitteln, insbesondere hinsichtlich der Geschwindigkeit, Genauigkeit und Zuverlässigkeit der Reaktionen auf Sicherheitsvorfälle. Die gewonnenen Erkenntnisse sollen dazu beitragen, Verbesserungspotenziale aufzuzeigen und die Implementierung von SOAR-Systemen in der Praxis zu unterstützen.
\end{enumerate}