\section{Fazit}
In der vorliegenden Arbeit wurde ein eigenes [deep learning?] Modell erstellt und zur Analyse eines medizinischen Korpus genutzt. Dazu wurden [X] Textabschnitte annotiert und zur \acf*{NER} und Relationship Extraction genutzt.

An dem Projekt waren keine Personen mit fachlichem - das heißt medizinischem - Hintergrund geteiligt. Auch wenn die Verlässlichkeit der Annotationen durch Metriken wie Cohen's Kappa gestützt werden, lässt sich dadurch nicht direkt die Qualität der Daten ableiten.

Im Laufe des Projektes ist es deutlich geworden, dass es bereits geeignetere, domänenspezifische Software zur Annotation im medizinischen Kontext gibt. Doccano verfügt über eine benutzerfreundliche Oberfläche und erlaubt das Labeling von Entitäten, ermöglicht jedoch keine Verknüpfung der Labels mit medizinischen Datenbanken beziehungsweise \ac{MeSH} Codes. In diesem Zusammenhang wurde für andere Datensätze in dieser Domäne, wie dem \textit{NCBI-Disease} oder \textit{BC5CDR}, PubTator\footcite{wei2013} verwendet.\footcite[vgl.][S.9]{dogan2014}\footcite[vgl.][S.4]{li2016}
