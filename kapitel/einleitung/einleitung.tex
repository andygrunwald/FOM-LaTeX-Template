\section{Einleitung}

\subsection{Einordnung in den Kontext}
\begin{itemize}
	\item Thema
	\item Kontext
\end{itemize}

\subsection{Anwendungsfelder}
\begin{itemize}
	\item Zielgruppe
	\item Mehrwert
\end{itemize}
\subsubsection{Google, Here, Apple Maps, OSM}
\subsubsection{Humanitäre Hilfe}
\subsubsection{Wissenschaftliche Flächenbeobachtung (Regenwald)}

% \subsection{Zielsetzung}
% Kleiner Reminder für mich in Bezug auf die Dinge, die wir bei der Thesis beachten sollten und \LaTeX{}-Vorlage für die Thesis.

% \subsection{Aufbau der Arbeit}
% Kapitel \ref{infos} enthält die Inhalte des Thesis-Days und alles, was zum inhaltlichen erstellen der Thesis relevant sein könnte. In Kapitel \ref{latexDetails} \nameref{latexDetails} findet ihr wichtige Anmerkungen zu \LaTeX{}, wobei die wirklich wichtigen Dinge im Quelltext dieses Dokumentes stehen (siehe auch die Verzeichnisstruktur in Abbildung \ref{fig:verzeichnisStruktur}).