\section{Einleitung}

Die Weltwetterorganisation hat in diesem Jahr einen neuen Bericht zu den Auswirkungen der
wetter- oder klimabedingten Naturkatastrophen in den Jahren 1970 bis 2019 veröffentlicht.
Die gravierendste Erkenntnis ist, dass die Anzahl der Stürme, Überschwemmungen, Dürren und
extremen Hitzeereignisse seit 1970 deutlich angestiegen ist.
Folgendermaßen hat sich die Zahl der Wetterextreme seit dem Jahr 1970 mehr als vervierfacht.
Ging es damals noch um etwa 71 Katastrophen pro Jahr, sind es heutzutage um die 316
wetterbedingten Extreme.
Auf der einen Seite ist zwar die Summe der Todesopfer seit den 1970ern gesunken,
der wirtschaftliche Schaden hat sich aber durch die Zunahme der Katastrophen verschlimmert.
	{\textcolor{red}{QUELLE}}
Diverse Frühwarnindikatoren unterstützen bei der frühzeitigen Evakuierung der gefährdeten
Menschen, aber das Eintreten der Katastrophe kann nicht verhindert werden.
Demzufolge passiert es dennoch, dass beispielsweise gesamte Städte überflutet werden und
komplette Straßen unbenutzbar sind. \\
Entscheidende Informationen über das Ausmaß der Zerstörung können in diesen Situationen
Satellitenbilder geben.
Durch deren Sichtweise auf die Unglücksorte lässt sich schnell und gezielt helfen,
beispielsweise indem zeitnah ersichtlich wird welche Zugangswege Hilfsorganisationen nutzen können.
Ziel dieser Projektarbeit ist es demnach, ein Modell zu erstellen,
welches eine semantische Segmentierung von Satellitenbildern vornimmt.
Zur Erreichung dieses Ziels wird ein Convolutional Neural Network erstellt.
Mit diesem Algorithmus kann ein Bild in verschiedene Objekte unterteilt und diese in entsprechende
Klassen eingeordnet werden.
Durch das Zuführen von neuen Bildern aus Katastrophengebieten in das entwickelte Modell sollen noch bestehende
Straßen und Gebäude zeitnah ermittelt werden.
Die Datengrundlage, die in dieser Ausarbeitung verwendet wird, wird vom
Defence Science and Technology Laboratory (Dstl) bereitgestellt.
Dieses beinhaltet 1km x 1km große Satellitenbilder. \\
Schlussendlich soll das Modell dem Humanitarian OpenStreetMap Team (HOT) zur Verfügung gestellt werden,
um deren globale Arbeit zu unterstützen.
Bei dieser Gemeinschaft geht es darum, Landkarten für gefährdete Regionen zu erstellen,
bei denen bisher nur wenige Daten zur Verfügung stehen. {\textcolor{red}{QUELLE}}
Mithilfe dieser Karten soll es Katastrophenhelfern ermöglicht werden die Bedürftigen schnell und gezielt
erreichen und versorgen zu können.
Im Zusammenhang mit Naturkatastrophen können das Katastrophenmanagement revolutioniert und Risiken verringert werden.


%\subsection{Einordnung in den Kontext}
%\begin{itemize}
%	\item Thema
%	\item Kontext
%\end{itemize}

%\subsection{Anwendungsfelder}
%\begin{itemize}
%	\item Zielgruppe
%	\item Mehrwert
%\end{itemize}
%\subsubsection{Google, Here, Apple Maps, OSM}
%\subsubsection{Humanitäre Hilfe}
%\subsubsection{Wissenschaftliche Flächenbeobachtung (Regenwald)}

% \subsection{Zielsetzung}
% Kleiner Reminder für mich in Bezug auf die Dinge, die wir bei der Thesis beachten sollten und \LaTeX{}-Vorlage für die Thesis.

% \subsection{Aufbau der Arbeit}
% Kapitel \ref{infos} enthält die Inhalte des Thesis-Days und alles, was zum inhaltlichen erstellen der Thesis relevant sein könnte. In Kapitel \ref{latexDetails} \nameref{latexDetails} findet ihr wichtige Anmerkungen zu \LaTeX{}, wobei die wirklich wichtigen Dinge im Quelltext dieses Dokumentes stehen (siehe auch die Verzeichnisstruktur in Abbildung \ref{fig:verzeichnisStruktur}).