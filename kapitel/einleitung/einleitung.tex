\section{Einleitung}

\subsection{Zielsetzung}

Die chronische spontane Urtikaria gehört zu der Gruppe chronischer Urtikaria Erkrankungen. Gekennzeichnet ist diese
durch das Wiederauftreten von Quaddeln und/oder Angioödemen über einen Zeithorizont von mehr als sechs
Wochen~\footcite[\vglf][\pagef 4]{savic.2020}
Die geschätzte weltweite Prävalenz chronischer Urtikaria Erkrankten beträgt schätzungsweise 1\%. Es liegt lediglich eine
geschätzte Prävalenz vor, da es Schwierigkeiten bei der Klassifizierung, der Identifizierung sowie der Diagnose der
Erkrankung gibt. Dies ist vor allem auf erhebliche Verzögerungen bei der Diagnose sowie unzureichende Kenntnisse über
die chronische Urtikaria zurückzuführen~\footcite[\vglf][\pagef 4]{savic.2020}\\

Die oben angeführte Problematiken wurden zum Anlass genommen ein Projekt zu initiieren, welches den Auftrag verfolgt
einen Beitrag zur bekämpfung der chronischen spontanen Urticaria Krankheit zu leisten. Begleitet wird das Projekt von
Ärzten und Spezialisten der Charité in Berlin.\\

Die vorliegende Hausarbeit behandelt das Teilprojekt ,,Information Retrieval''. Ziel dieses Teilprojektes war es aus einem
Text-Corpus mit insgesamt über 500 medizinische Fachartikel automatisiert Informationen aus den Texten zu extrahieren,
um das Wissen über die Krankheit, erfolgreiche Behandlungsmöglichkeiten etc. zu erweitern.\\

\subsection{Aufbau der Arbeit}
Zunächst betrachten wir die theoretischen Hintergründe des \acf{NLP} und der \acf{NER} um die Grundlage für den zweiten Teil der Arbeit zu schaffen. Dabei gehen wir in Kapitel \ref{sec:TextPreprocessing} auf die Vorverarbeitung des Textkorpus ein, um dann im Kapitel \ref{sec:NER} auf die Typen und Aufgaben von \acf{NER} einzugehen.

Im Praxisteil der Arbeit (\ref{sec:Praxis}) gehen wir dann auf die Erstellung eines eigenen Modells zur \acf{NER} ein. Dabei beschreiben wir den Prozess der Annotierung (\ref{sec:Labeling} bis \ref{sec:Software}) und des Trainings um dann eine Evaluierung des Annotierungsprozesses (\ref{sec:Iaa}) und des Trainings (\ref{sec:ModelTraining}) vorzunehmen.

Im abschließenden Teil (\ref{sec:Fazit}) ziehen wir ein Fazit über das Projekt und fassen die wesentlichen Erkenntnisse der Erstellung des domänenspezifischen \ac{NER}-Modells und der Analyse von Beziehungen zwischen Entitäten zusammen. Weiter blicken wir auf mögliche Verbesserungspotentiale des gewählten Vorgehens und darüber hinaus zusätzliche Analyseaspekte die von uns nicht berücksichtigt wurden.
% Kapitel \ref{infos} enthält die Inhalte des Thesis-Days und alles, was zum inhaltlichen Erstellen der Thesis relevant sein könnte.
% In Kapitel \ref{latexDetails} \nameref{latexDetails} findet ihr wichtige Anmerkungen zu \LaTeX{}, wobei die wirklich wichtigen Dinge im Quelltext dieses Dokumentes stehen (siehe auch die Verzeichnisstruktur in Abbildung \ref{fig:verzeichnisStruktur}).

