\section{Einleitung}

\subsection{Zielsetzung}

Die chronische spontane Urtikaria gehört zu der Gruppe chronischer Urtikaria Erkrankungen. Gekennzeichnet ist diese
durch das Wiederauftreten von Quaddeln und/oder Angioödemen über einen Zeithorizont von mehr als sechs
Wochen~\footcite[\vglf][\pagef 4]{savic.2020}
Die geschätzte weltweite Prävalenz chronischer Urtikaria Erkrankten beträgt schätzungsweise 1\%. Es liegt lediglich eine
geschätzte Prävalenz vor, da es Schwierigkeiten bei der Klassifizierung, der Identifizierung sowie der Diagnose der
Erkrankung gibt. Dies ist vor allem auf erhebliche Verzögerungen bei der Diagnose sowie unzureichende Kenntnisse über
die chronische Urtikaria zurückzuführen~\footcite[\vglf][\pagef 4]{savic.2020}\\

Die oben angeführte Problematiken wurden zum Anlass genommen ein Projekt zu initiieren, welches den Auftrag verfolgt
einen Beitrag zur bekämpfung der chronischen spontanen Urticaria Krankheit zu leisten. Begleitet wird das Projekt von
Ärzten und Spezialisten der Charité in Berlin.\\

Die vorliegende Hausarbeit behandelt das Teilprojekt ,,Information Retrieval''. Ziel dieses Teilprojektes war es aus einem
Text-Corpus mit insgesamt über 500 medizinische Fachartikel automatisiert Informationen aus den Texten zu extrahieren,
um das Wissen über die Krankheit, erfolgreiche Behandlungsmöglichkeiten etc. zu erweitern.\\

\subsection{Aufbau der Arbeit}

Kapitel \ref{infos} enthält die Inhalte des Thesis-Days und alles, was zum inhaltlichen Erstellen der Thesis relevant sein könnte.
In Kapitel \ref{latexDetails} \nameref{latexDetails} findet ihr wichtige Anmerkungen zu \LaTeX{}, wobei die wirklich wichtigen Dinge im Quelltext dieses Dokumentes stehen (siehe auch die Verzeichnisstruktur in Abbildung \ref{fig:verzeichnisStruktur}).

