\newpage
\section{Umsetzung von Testautomatisierung}

\subsection{Strukturelle Voraussetzungen}
Im Gegensatz zu der vorherigen Umsetzung, die den Prinzipien der prozeduralen Programmierung folgte, fand bei der Neuentwicklung der Schnittstelle die beschriebenen und etablierten Best-Practices Anwendung. Dies hat den großen Vorteil, dass die erstellte Klassenstruktur konsequent dem Single Responsiblity Prinzip erfüllen. Die Verzeichnis und zeitgleich verwendete Namespace Struktur der Klassen ist wie folgt aufgebaut:

\begin{itemize}
	\item Commands
	\item Config
	\item Exporter
	\item Importer
	\item Mapper
	\item Reader
	\item Resource
	\item Utils
	\item Writer
\end{itemize}

In den verschiedenen Verzeichnissen befinden sich einzelne Klassen die für eine bestimmte Aufgabe konzipiert sind. Im Bereich Commands existiert zum Beispiel eine EtosExportOrders und eine EtosImportArticles Klasse.

---- Weitere Beschreibung wie es aussieht. Abhängigkeiten sind via Dipendency Injection und constructor geladen worden.

\subsection{Auswahl der Testwerkzeuge}
\subsubsection{PHP-Unit}
Die Auswahl für die Durchführung und Erstellung der benötigten Unit- und Integrationstest viel auf das Framework PHPUnit. Dies ist in der PHP-Entwicklung der defakto Standard für diese Aufgaben geworden.

PHPUnit ist ein Testframework welches besonders für automatisiertes Testen einzelner Einheiten (Units, meist Klassen oder Methoden) geeignet ist. PHPUnit wird ab
der Version 3.0 unter der Berkeley Software Distribution (BSD)-Lizenz veröffentlicht und kann problemlos als PHP-Archive (phar) Paket heruntergeladen und installiert
werden.

PHPUnit basiert auf dem xUnit-Konzept, welches auch Anwendung in anderen Programmiersprachen findet (z.B. JUnit für Java).

\subsubsection{Behat/Mink}
- GUI Testing
- Gute Abstraktion der zu Testenden Szenarion


\subsubsection{Humbug}
- einziges Tool was für PHP Mutation Testing durchführen kann.


\subsection{Technische Umsetzung}
\subsubsection{Unit-Tests}
Für die Umsetzung von Unit-Tests ziehen wir exemplarisch die Config und Reader Klassen heran


\subsubsection{Integration}

Mapper Klasse

\subsubsection{Mutation-Testing}

Durchführung:
\begin{lstlisting}[caption=Ausführung des Mutation-Testing]{MutationTesting}
Humbug running test suite to generate logs and code coverage data...

19 [==========================================================] 18 secs

Humbug has completed the initial test run successfully.
Tests: 19 Line Coverage: 71.16%

Humbug is analysing source files...

Mutation Testing is commencing on 22 files...
(.: killed, M: escaped, S: uncovered, E: fatal error, T: timed out)

SSSSSSSSSS..S.......S...................MMMMMMMM.....M...... |   60 (15/22)
.M.SSSSSSSS..M.MMMMMMMM.MSS....MMMMMMMMMMM..MM.M.MMMMMMMM..M |  120 (15/22)
MMMSSSSSSSSSSSSSSSSSSSSSSSSSSSSSSSSSSSSSSSSSSSSSSSSSSSSSSSSS |  180 (15/22)
SSSSSSSSSSSSSSSSSSSSSSSSSSSSSSSSSSSSSSSSSSSSSSSSSSSSSSSSSSSS |  240 (15/22)
SSSSSSSSSSSSSSSSSSSSSSSSSSSSSSSSSSSSSSSSSSSSSSSSSSSSSSSSSSSS |  300 (15/22)
SSSSSSSSSSSSSSSSSSSSSSMMMMMMMMSSMMSMMMMMMMMSSSMMMSSMMMMMMM.. |  360 (18/22)
TM.M..M

367 mutations were generated:
60 mutants were killed
229 mutants were not covered by tests
77 covered mutants were not detected
0 fatal errors were encountered
1 time outs were encountered

Metrics:
Mutation Score Indicator (MSI): 17%
Mutation Code Coverage: 38%
Covered Code MSI: 44%
\end{lstlisting}



\subsection{Automatische Durchführung der Tests via continuous integration}