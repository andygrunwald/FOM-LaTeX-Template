\newpage
\section{Praxis} \label{latexDetails}
\subsection{Erstellung eines domänenspezifischen NER Modells}
Wie im Kapitel \ref{NER} erläutert lässt sich zwischen allgemeinen und domänenspezifischen \ac{NER}-Modellen unterscheiden. Domänenspezifische Modelle sind dabei solche, die für einen dedizierten Themenbereich erstellt wurden und daher in diesem Kontext besonders gute Ergebnisse erzielen.
Es gibt bereits spezialisierte Modelle im Bereich der Medizin die mit einem medizinischen Korpus trainiert wurden. Dazu zählen zum Beispiel BioBERT, ScispaCy oder Y\footcite[S.12]{li2020}. Im folgenden beschreiben wir, wie wir ein eigenes Modell für das Themengebiet der Urtikaria-Forschung erstellt haben.
\subsubsection{Labeling von Trainingsdaten}
Das annotieren von Trainingsdaten hat einen zentralen Anteil bei der Erstellung eines eigenen Modells. Akkurate Daten sind essentiell für die Genauigkeit des resultierenden Modells. Neben der Menge der zum Training zur Verfügung stehenden Daten ist es unterlässlich, dass diese widerspruchsfrei sind.

\subsubsection{Annotierungsrichtlinien}
\footcite[vgl.][]{neves2014}

\subsubsection{Software}
\footcite[vgl.][]{neves2014a}

\subsubsection{Training des Modells}

\subsubsection{Evaluierung des Modells}
\footcite[]{tsai2006}

\subsection{Markeranalyse}
\subsection{Überführung in Tabellenstruktur}
\subsection{Markerkorrelationen}
\subsection{Aufstellung des Gradingsystems}
